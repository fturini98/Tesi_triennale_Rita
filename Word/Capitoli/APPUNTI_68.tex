I decenni del secondo dopoguerra furono descritti dallo storico Eric John Ernest Hobsbawm come gli anni di una \enquote{straordinaria crescita economica e di trasformazione sociale, che probabilmente hanno modificato la società umana più profondamente di qualunque altro periodo di analoga brevità} \footcite{Hobsbawm}.
Questa definizione risulta particolarmente adeguata per una nazione come l'Italia che, coinvolta da rivoluzioni economiche, sociali e antropologiche, ne uscì completamente mutata.



\subsection{2 I giovani vogliono un cambiamento}
La profonda trasformazione della società e l'immobilismo istituzionale scaturirono contestazioni e movimenti popolari a cavallo tra gli anni '60 e '70.
Tra i protagonisti di questi anni c'è il movimento studentesco.
Gli studenti, sempre più numerosi\footnote{Secondo i dati Istat \url{https://www.istat.it/it/files//2019/03/cap_7.pdf} 
nell'anno accademico 1967-68 gli studenti universitari erano 500 mila contro i 268 mila del 1960-61, il numero di studentesse in particolare era più che raddoppiato}, si ribellarono a un sistema scolastico inadeguato, senza prospettive lavorative e classista.
I giovani si appropriarono di linguaggi, stili, miti e simboli propri e peculiari e riuniti in assemblee iniziarono a discutere sul loro ruolo sociale e a richiedere un maggiore potere negli organi decisionali, la riforma del curriculum, la diminuzione delle tasse, la liberalizzazione del piano di studi, un approccio critico al sapere e l’educazione alla libera discussione.
Si fecero portavoce del rifiuto dell'autoritarismo, delle autorità, sia del governo sia familiari, e dei valori della nuova società consumistica che privilegiava la massa danneggiando il singolo.
Rappresentativo del rifiuto all'omologazione di massa dei giovani all'interno delle università è l'intervento che Mario Savio\footnote{Mario Savio fu uno studente e attivista statunitense, personaggio chiave e leader del Free Speech Movement, con il suo discorso a Berkley nell'ottobre 1964 diventò il simbolo del movimento degli studenti} fece a Berkley nel 1964 nel quale invitò i suoi compagni a non partecipare passivamente al processo di automatizzazione in cui erano coinvolti.
Lo fa con queste parole: \enquote{and if President Kerr in fact is the manager, then I tell you something - the Faculty are a bunch of employees! And we're the raw material! But we're a bunch of raw materials that don't mean to have any process upon us, don't mean to be made into any product, don't mean to end up being bought by some clients of the University, be they the Government, be they industry, be they organized labor, be they anyone! We're human beings!}\footcite{Savio}.
Icastica fu anche Giuditta Pieti che riflettendo sulla condizione studentesca e le sue esigenze nelle colonne di \textit{Il Giacobino} scrisse nel 1966 una riflessione simile: \enquote{Il rendersi conto che la situazione attuale della società ostacola l’esplicarsi delle capacità di quei giovani […], porta coloro che sono più sensibili a quest'istanza, a chiedersi cosa si può fare, come ci si può opporre a un inglobamento entro schemi precostituiti per non correre il rischio di diventare degli elementi facilmente sostituibili di un ingranaggio} \footcite{Pieti}.

Sono i giovani a farsi portavoce di una protesta che non è esclusivamente contro la condizione studentesca, ma contro la società che si era formata negli ultimi anni.
Gli studenti si fecero portavoce del rifiuto dell'autoritarismo, delle autorità, sia del governo sia familiari, e dei valori della nuova società consumistica che privilegiava la massa danneggiando il singolo e si misero a capo di un movimento che voleva scardinare i tabù concernenti la sessualità.

Furono dei liceali, del Liceo Parini di Milano, che anticiparono il discorso pubblico e la politicizzazione della liberalizzazione sessuale pubblicando nel 1966 nel loro giornale \textit{La Zanzara} un'inchiesta \textit{Che cosa pensano le ragazze d’oggi}.
Al centro del sondaggio ci sono il divorzio, la contraccezione, di cui era vietato discutere pubblicamente, l'assenza di educazione sessuale: fu uno scandalo nazionale. 
Le ragazze della nuova generazione vivevano più liberamente la sessualità, senza sensi di colpa morali conseguenti alle idee propagate dall'etica cattolica, volevano rapporti prematrimoniali grazie all'uso di contraccettivi e un futuro lavorativo non all'interno delle mura domestiche: \enquote{Non vogliamo più un controllo dello stato e dalla società sui problemi del singolo e vogliamo che ognuno sia libero di fare ciò che vuole, a patto che ciò non leda la libertà altrui. Per cui, assoluta libertà sessuale e modifica totale della mentalità”} \footcite{Zanzara}.
I giovani per primi vedono le contraddizioni della nuova Italia.
È questa nuova generazione che denuncia una società che si mostra sempre più libera, ma in cui in realtà prevale ancora un forte moralismo.

È ancora aperto il dibattito sul ruolo storico del '68 considerandone luci e ombre. 
Le rivendicazioni dei ragazzi di libertà e l'autonomia rispetto alle istituzioni tradizionali da molti interpretate come indizio di una nuova consapevolezza, della modernità, di rottura con il mondo antico per alcuni erano invece piene di contraddizioni.
Tra i personaggi pubblici che analizzarono questi anni fu Pasolini che sentendo il peso del suo ruolo di intellettuale riteneva necessario intervenire e smascherare quella che per lui era una falsa rivoluzione.
Lo scrittore denunci la complicità del movimento del Sessantotto con i processi di omologazione della modernità, gli studenti non riuscendo a ribellarsi davvero finiscono per diventare loro stessi strumenti del Capitale.
Pasolini non condanna integralmente i movimenti studenteschi, ne fa una lettura complessa.
Da una parte concorda con la necessità di una rivoluzione, ma dall'altra non può non considerare che i giovani contestatori convinti della loro lotta stiano in realtà agendo in nome del capitalismo che a loro insaputa li stava utilizzando .
Il sistema, infatti, dando la parvenza di concedere libertà assimilava ogni contestazione, integrava ogni possibile azione non conforme al sistema  per annullarla.
È una ribellione guidata dall'alto e i ragazzi ingenuamente si sentono di esserne a capo.

Politicamente la contestazione studentesca si risolse in un fallimento, forse questo a conferma di ciò che Pasolini affermava, ma non si può negare il ruolo centrale che ricoprì nella lotta per la liberalizzazione dei corpi e dei costumi. 

