\section{Una nuova Italia}
\enquote{l'Italia si è trasformata tanto da risultare quasi irriconoscibile. Intanto essa è divenuta una delle nazioni economicamente più forti del mondo (...) bruciando la tappe di un processo straordinariamente rapido di accumulazione, di urbanizzazione, di secolarizzazione. Le culture contadine dei secoli precedenti non sono scomparse del tutto, ma sono state sostituite da un'unica cultura nazionale urbana. (...) Durante gli anni dello Stato repubblicano, insomma, l'Italia ha assistito al più profondo rivolgimento sociale della sua storia}\footcite{Ginsborg7}, queste sono le parole di Paul Ginsborg per descrivere, brevemente, il veloce progresso che coinvolse il nostro Paese.
L'Italia che oggi conosciamo ha subito una trasformazione traumatica nel giro di quarant'anni.


\subsection{Il trauma italiano}
La penisola italiana fu a lungo un paese prevalentemente agricolo, di antiche tradizioni e costumi, ma dal secondo dopoguerra la situazione era destinata a cambiare.
L'improvviso sviluppo economico e industriale, il cosiddetto \enquote{miracolo economico}, la crescita della ricchezza, il forte movimento migratorio\footnote{a cavallo tra gli anni '50 3 gli anni '70 più di 9 milioni di italiani emigrarono} che rimescolò la popolazione, il dilagare di nuovi oggetti di consumo e mezzi di comunicazione sono alcune delle ragione per cui l'Italia cambiò volto.
La trasformazione a cui fu sottoposta non avvenne senza contraddizioni, fu un processo troppo rapido e radicale per poter coinvolgere l'intero Paese in modo omogeneo e permettere a chi ci abitava di metabolizzarlo in modo consapevole.

Il vecchio mondo perdurato per secoli improvvisamente si frantumò sotto la spinta di una nuova fase storica.
È la nuova era del consumismo, dell'industrializzazione e del neo-capitalismo, è un'onda di cambiamenti che travolse il vecchio mondo senza assimilarlo.
Le campagne vennero abbandonate e le periferie tentarono di svilupparsi imitando, con esiti destabilizzanti, le nuove metropoli.
Non è infatti possibile trapiantare nelle periferie e nelle vite di persone cresciute secondo i valori della vita contadina quei modi di vivere e idee dei grandi centri senza che esse risultassero \enquote{strane}\footnote{ci si può avvalere per questo del concetto di \textit{misplaced ideas}}.
L'Italia si divise sempre di più in zone con velocità di trasformarsi diverse.
Le ideologie che si propagavano sempre di più nei centri industriali e tra i giovani che sempre più frequentavano l'università facendo il loro ingresso nei mondi più arretrati causarono veri e propri shock culturali.

Gli italiani cercavano il loro posto nella nuova società, tentavano di essere moderni senza però sapere come fare e come adattarsi al cambiamento, sono destabilizzati.
I giovani che si sono spostati verso il Nord o verso il centro delle metropoli si trovano disorientati,  è una generazione sradicata dall'antica tradizione, senza punti di riferimento fissi.
I modelli antichi sono ormai caduti, inattuali, e quelli moderni troppo distanti dalla cultura tradizionale per essere assimilati in modo costruttivo e non traumatico.
\\Pier Paolo Pasolini si riferisce a questo processo con questi termini: \enquote{il trauma italiano del contatto tra "l'arcaicità" pluralistica e il livellamento industriale ha forse un solo precedente: la Germania prima di Hitler. Anche qui i valori sono stati distrutti dalla violenta omologazione dell'industrializzazione: con la conseguente formazione di quelle enormi masse, non più antiche (contadine, artigiane) e non ancora moderne (borghesi). (...) Non siamo più di fronte, come tutti sanno, a "nuovi tempi", ma a una nuova epoca della storia umana: di quella storia umana le cui scadenze sono millenaristiche. Era impossibile che gli italiani reagissero peggio di così}.\footcite{Scritti7} 

\paragraph{}Mentre l'Italia procedeva nella sua trasformazione e gli italiani goffamente cercavano di stare al passo lo Stato e le sue istituzioni non si mostrarono pronte a gestire tali cambiamenti.
Pier Paolo Pasolini parla addirittura di un \enquote{drammatico vuoto di potere}\footcite{Scritti7}.
I politici italiani sembravano non capire ciò che negli ultimi anni era cambiato e invece di controllare il nuovo potere consumistico lo servivano.
\\Il Partito Democristiano, che in questi anni deteneva la maggioranza, forse inconsapevole della profondità del cambiamento in atto e speranzoso che gli equilibri non mutassero in modo eccessivo si dimostrò inadeguato nel guidare il Paese.
L'eccessivo ottimismo, fallimentare, di poter riuscire ad amministrare e tenere controllato il progresso che investiva l'Italia fece in modo tale che gli italiani si ritrovassero senza guida e dovessero cercare autonomamente la strada per trovare la loro nuova identità ribellandosi alle istituzioni dalle quali non si sentivano più rappresentati.
\\Neanche il Partito Comunista sembrò all'altezza delle aspettative dei suoi elettori.
Enrico Berlinguer era: \enquote{il nome che il sottoproletariato toscano dà a un progetto politico inteso a gestire in senso possibilmente civile, razionale e solidale quell'ultima fase della storia italiana; il progetto di conciliare quel che restava del vecchio mondo agricolo con le nuove tendenze della società post-agricola e post-industriale}\footcite{Brugnolo}.
Il compromesso storico si presentò come l'ultima occasione per coinvolgere gli strati popolari nella riforma della società.
L'immobilismo e il fallimento del Pci nel farsi guida di questa rivoluzione deluse ogni speranza.
\\Queste sono le prerogative che portarono a una grande stagione di azione collettiva: l'inerzia delle istituzioni fu sostituita dall'attività del popolo.

\paragraph{}I giovani presto si resero conto di non voler prendere parte al progetto di omologazione che stava travolgendo l'Italia.
\\Giuditta Pieti già nel 1966 espresse le esigenze degli studenti sulle colonne di \textit{Il Giacobino}:\enquote{Il rendersi conto che la situazione attuale della società ostacola l’esplicarsi delle capacità di quei giovani […], porta coloro che sono più sensibili a quest'istanza, a chiedersi cosa si può fare, come ci si può opporre a un inglobamento entro schemi precostituiti per non correre il rischio di diventare degli elementi facilmente sostituibili di un ingranaggio} \footcite{Pieti}\footnote{è questo un discorso che riprende le parole di Marco Savio simbolo del movimento degli studenti \enquote{and if President Kerr in fact is the manager, then I tell you something - the Faculty are a bunch of employees! And we're the raw material! But we're a bunch of raw materials that don't mean to have any process upon us, don't mean to be made into any product, don't mean to end up being bought by some clients of the University, be they the Government, be they industry, be they organized labor, be they anyone! We're human beings!} \url{ https://www.youtube.com/watch?v=0RjqxIO87_s&feature=youtu.be}}.

I movimenti studenteschi del '68 si fecero portavoce del rifiuto dell'autoritarismo, delle autorità, sia del governo sia familiare, e dei valori della nuova società consumistica che privilegiava la massa danneggiando il singolo.
La protesta non fu solo contro la condizione studentesca, ma contro la nuova realtà che si stava delineando.

\paragraph{}È ancora aperto il dibattito sul ruolo storico del '68 considerandone luci e ombre. 
Le rivendicazioni dei ragazzi di libertà e l'autonomia rispetto alle istituzioni tradizionali da molti interpretate come indizio di una nuova consapevolezza, della modernità, di rottura con il mondo antico, per alcuni erano invece piene di contraddizioni.

Tra i personaggi pubblici che analizzarono questi anni fu Pasolini che, sentendo il peso del suo ruolo di intellettuale, ritenne necessario intervenire e smascherare quella che per lui era una falsa rivoluzione.
Pasolini non condanna integralmente i movimenti studenteschi, ne fa una lettura complessa.
Lo scrittore denunciò la complicità del movimento del Sessantotto con i processi di omologazione della modernità, gli studenti non riuscendo a ribellarsi davvero finirono per diventare loro stessi strumenti del Capitale.
Pur condividendo con loro  la necessità di una rivoluzione, non poté non considerare come i giovani contestatori convinti della loro lotta stessero in realtà agendo in nome del capitalismo che a loro insaputa li stava utilizzando.
Il sistema, infatti, dando la parvenza di concedere libertà assimilava ogni contestazione, integrava ogni possibile azione non conforme al sistema per annullarla.
\\È una ribellione guidata dall'alto e i ragazzi ingenuamente si sentono di esserne a capo.

Nonostante i movimenti di contestazione giovanile abbiano contribuito ad accelerare il processo di svecchiamento dei costumi nella società e promosso numerose lotte per l’acquisizione di diritti civili, già nei mesi successivi alla fine delle manifestazioni apparve chiaro che molte delle battaglie portate avanti dai giovani in rivolta non avrebbero prodotto gli effetti utopici desiderati.
In particolare, nel campo della sessualità, il tentativo di emancipare il desiderio dalla repressione fallì.
Il famoso slogan \textit{Godetevela senza freni} rappresentò non un’effettiva fuoriuscita dal meccanismo capitalista, ma la sua più evoluta espressione.

Politicamente la contestazione studentesca si risolse in un fallimento.
La visione di Pasolini del '68 come una falsa rivoluzione strumento del Capitale sembra diventare realtà.

\paragraph{} L'Italia stava cambiando, l'economia si era rivoluzionata, la tradizione culturale cercava di modernizzarsi.
A rimanere antico era il sistema legislativo.
\\Il codice di diritti e doveri dei coniugi rimaneva invariato sullo stampo del Codice Pisanelli\footnote{Il Codice Pisanelli risale al 1865} che conservando il marito a capo della famiglia, la patria potestà, l'indissolubilità del matrimonio ed eliminando solo la necessità dell'autorizzazione maritale per ogni transizione economica, manteneva profondamente radicata la famiglia patriarcale nella quale l'uomo aveva il controllo e poteva mantenerlo anche facendo uso della violenza.
A livello penale rimase in vigore il Codice Rocco\footnote{Il Codice Rocco risale al 1930}, codice fascista che considerava la violenza carnale un reato morale e la contraccezione e l'aborto reati contro la stirpe.
Si creò così uno scenario carico di contraddizioni non risolte.
L'Italia che si mostrò moderna con l'apertura al voto alle donne non aveva finito di fare i conti con il passato.

Le grandi trasformazioni che avevano investito l'Italia, tra cui il boom economico, la decrescita dell'agricoltura, l'emigrazione, mandarono in crisi l'organizzazione gerarchica e autoritaria della famiglia.
Dal Sessantotto in poi molte furono le battaglie per modificarne la struttura\footnote{Ginsborg considera fondamentale lo studio dell'istituzione familiare per rileggere la storia nazionale italiana} e togliere la donna dall'oppressione esercitata dal marito.
Si aprì una stagione di grande fermento e di lunghe battaglie per tutelare l'autodeterminazione e perché venisse attuata una riforma del diritto di famiglia non più adatto al nuovo modo di pensare e vivere la famiglia, le relazioni di coppia e la sessualità.
\paragraph{}Il corpo entrò nell'agenda politica italiana costruendo una nuova geografia delle relazioni sociali.
Si capovolse il rapporto tra sesso femminile, famiglia e società, venne meno la netta separazione tra l'uomo \enquote{breadwinner}, produttore del reddito familiare e la donna dedita al focolare domestico.
Al suo posto iniziò ad affermarsi il nuovo modello di dual-breadwinner che aprì la strada, ancora lunga, alla parità di genere in ambito economico-lavorativo.

Il 1970 fu l'anno in cui sembrò concludersi la lunga lotta\footnote{La prima proposta di legge risaliva al 1965 e fu avanzata dal socialista Loris Fortuna che propose un testo moderato che limitava questo diritto ad alcune situazioni definite ma la Dc bloccò l'iter parlamentare, nel 1969 vennero fatte alcune modifiche, il Pci, dopo lunghe discussioni ed essere giunto alla conclusione di essere favorevole al divorzio e al riconoscimento dei mutamenti avvenuti nella società ma di voler anche difendere la famiglia la cui funzione di stabilizzatrice sociale non doveva essere messa in discussione, assicurava il suo appoggio mentre la Chiesa invitava i suoi fedeli a pregare per allontanare questa possibilità} per introdurre il divorzio, era uno storico obiettivo del femminismo.
La legge per il divorzio venne riconfermata, qualche anno dopo, dalla vittoria del no al referendum per l'abrogazione\footnote{Al referendum del 1975 vinse il no con il 59\% dei voti \url{https://www.istat.it/it/files//2019/03/cap_9.pdf}}

Da questo momento in poi il movimento femminista si dedicò alla battaglia a favore dell'aborto. L'interruzione volontaria di gravidanza pur essendo illegale era un fenomeno diffuso: secondo uno studio del 1961 firmato dalla giornalista Milla Pastorino\footcite{Pastorino} ogni 100 gravidanze portate a termine 50 erano interrotte.
Era necessario che la questione dell'aborto diventasse pubblica.

Il movimento  si dedicò non solo a diffondere il principio di autodeterminazione, a rilanciare il tema della contraccezione e predisporre a livello nazionale iniziative per favorire una consapevole gestione della propria sessualità, ma si impegnò nell'offrire assistenza alle donne che avevano bisogno organizzando viaggi verso le città estere dove l'aborto era regolamentato, a creare centri dove effettuare visite ginecologiche effettuate da medici volontari.
Alcuni gruppi decisero inoltre di fondare anche nuclei di autogestione dell'aborto dopo aver imparato il metodo dell'aspirazione agendo clandestinamente e assumendosi la responsabilità clinica e penale.

Nel maggio 1978 vennero approvate le \enquote{Norme per la tutela sociale della maternità e sull'interruzione volontaria di gravidanza}, poi confermata da un referendum popolare nel 1981.
Questa legge fu una grande conquista, l'aborto non era più reato, ma lasciò una parte del movimento femminista con l'amaro in bocca poiché l'autodeterminazione delle donne non venne tutelata: prima di abortire era d'obbligo consultarsi con un medico e un assistente sociale e poi aspettare una settimana di \enquote{meditazione} prima di poter essere sottoposte all'intervento, le ragazze sotto la maggiore età dovevano avere il permesso dei genitori e infine veniva riconosciuto ai medici il diritto all'obiezione di coscienza.

Nel 1975 la riforma del diritto di famiglia investì un ampio spettro di questioni come il matrimonio, la filiazione, le violenze sessuali.
Dopo anni di lotte per la parità venne affermata l'uguaglianza tra i coniugi che sposandosi assumevano gli stessi diritti e doveri e abolita la figura del capofamiglia.
La rivoluzione sessuale introdusse nelle relazioni di coppia una novità: in esse doveva essere presente l'intesa affettiva, romantica ed erotica.
Impegnarsi in una relazione è una scelta, non un lavoro come esplicita chiaramente lo slogan: \enquote{Il matrimonio non è una carriera!}.
È un'unione basata sui valori del consenso e delle parità.

Altri importanti traguardi furono raggiunti a fine secolo quando nell'agosto 1981 il Parlamento approvò l'abrogazione della rilevanza penale della causa d'onore e del matrimonio riparatore e solo dal 1996 la violenza sessuale divenne reato contro la persona e non contro la morale.
Sono vittorie molto significative per i diritti delle donne e per la lotta contro la violenza.
Le ragazze non dovevano più essere percepite come oggetti di possesso maschile.

\paragraph{}Il femminismo cambiò il rapporto tra società e politica incoraggiando la politicizzazione di ampi gruppi sociali e di problemi culturali.
È attraverso queste nuove leggi che le istituzioni tentarono di riformarsi e di riequilibrare le relazioni all'interno del nucleo familiare dopo la grande trasformazione culturale che aveva investito il modo di pensare degli italiani, soprattutto delle donne, non più intenzionate a sottostare alle scelte di nessuno.
Con l'approvazione del divorzio e dell'aborto e la condanna alla violenza la donna si riappropriò del suo corpo e della sua vita.

I percorsi che portano a una trasformazione così netta della società, prima a livello di coscienza personale e poi a quello legislativo, sono lunghi, discontinui, mutano, accolgono nuove lotte al loro interno e a volte devono scendere a compromessi con un'élite politica non in grado di gestire i mutamenti, ancorata al passato.
La rivoluzione che vuole portare a una completa parità dei generi e a una totale libertà sessuale è una battaglia che non può ritenersi conclusa.
La problematica dell'identità e del rapporto tra sesso, genere e potere è oggi più che mai presente e va combattuta ogni giorno, a casa, sul lavoro, per strada, perché ogni azione è politica. 




\subsection{ La rivoluzione culturale }
L'Italia degli anni '70 e '80, degli \enquote{anni di piombo}\footnote{L'espressione deve la sua consacrazione al film del 1981 "Die bleierne Zeit" di Margarethe Von Trotta} investita dalla violenza dello stragismo e del terrorismo, fu attraversata anche da lotte, guidate soprattutto dai nuovi movimenti femministi, per l'emancipazione familiare e sociale della donna.
\\Al centro delle nuove discussioni si trovava il corpo come nuovo elemento di interesse collettivo.
Temi riguardanti la donna, il suo corpo e il suo ruolo nella vita del Paese, entrarono nel dibattito pubblico, il personale divenne politico.


Nel corso di questi decenni chiave nel processo di trasformazione culturale il vuoto di potere politico fu riempito dai modelli diffusi dal consumismo.
Il cinema, la televisione e i recenti canali di aggregazione e socializzazione come le culture giovanili e femministe avrebbero infatti modellato la nuova Italia.
Lo storico Silvio Lanaro parla di una società che si sarebbe autoriformata\footcite{Balestracci2}.


\subsubsection{La seconda ondata di femminismo}
Il femminismo del secondo Novecento riprese una lotta che risale alla fine del Settecento.
I primi tentativi di riconoscimento dei diritti delle donne portano infatti la data 1791 quando Olympe de Gouges pubblicò una \textit{Dichiarazione dei diritti delle donne} chiedendo l'estensione dei diritti universali dell'uomo.
Un anno dopo Mary Wollstonecraft provocò scalpore in Inghilterra con la sua \textit{Rivendicazione dei diritti delle donne}.
L'idea principale della Wollstonecraft era che l'oppressione a cui erano sottoposte le donne non fosse un fatto di natura bensì di cultura ed educazione.

La costruzione di un maschile e di un femminile con specifiche attribuzioni stereotipate fin dalla nascita venne messa in discussione e criticata dai movimenti femministi.
È questa una riflessione che verrà ripresa proprio in questi anni chiave per esempio da Elena Gianini Belotti in \textit{Dalla parte delle bambine}\footnote{saggio sociologico e pedagogico edito da Feltrinelli nel 1973}.
\\In questi anni si svilupparono studi interessanti sull'influenza del condizionamento sociale e culturale nella formazione del ruolo femminile dalla prima infanzia.
Secondo questa corrente di pensiero, la società preparerebbe fin da subito le bambine a diventare donne adatte a una società patriarcale, dedite alla cura della casa e della famiglia, sempre in ordine e obbedienti.
\\È attraverso i primi giochi, i primi commenti su come sia necessario comportarsi da "signorine" fino alle lezioni di Educazione domestica\footnote{Economia domestica divenne materia di insegnamento della scuola media a partire dalla Riforma Gentile dal 1963 si inserì nell'insegnamento di Applicazioni Tecniche ancora differenziato tra maschi e femmine, solo dal 1977 con in nuovo appellativo di Educazione Tecnica non si diversificò più in base al sesso dello studente} che in modo passivo le bambine e poi ragazze imparavano quale fosse il loro posto e ruolo nella società.
Ancora oggi si continua ad attribuire ai bambini le caratteristiche considerate tipiche del sesso di appartenenza invece di dare la possibilità a ognuno di crescere e sviluppare le proprie personali attitudini.

L'educazione è quindi una parte fondamentale nel processo di emancipazione, è necessaria perché le ragazze acquistino consapevolezza del ruolo sociale in cui sono state relegate e perché abbiano le competenze per liberarsene


\paragraph{}Essenziali per il nuovo sviluppo del movimento femminista furono alcune letture internazionali come il testo \textit{Il secondo sesso} (1949) di Simone de Beauvoir, edito in Italia solo dal 1963.
Anche questa autrice si interrogò sulle condizioni socio-culturali che nella storia contribuirono a relegare la donna in una posizione di inferiorità.
Secondo il suo punto di vista l'essere donna non è un destino psicologico o biologico, ma il risultato di una costrizione sociale.
Un processo culturale, di cui le donne sarebbero complici non ribellandosi, perpetuato nella storia avrebbe reso la figura femminile inferiore e dipendente dalla maschile.
Cercando di individuare le possibili cause della subordinazione e giudicando ogni individuo, uomo e donna in quanto coscienza, sostanzialmente libero invitava a una rifondazione teorica del femminismo e a un'unione delle donne, consapevoli della propria condizione, per combattere insieme le disuguaglianze.
L'obiettivo ultimo era la parità di diritti, di dignità e opportunità sociali, politiche ed economiche.

Le studentesse degli anni '60 e '70 furono tra la prime a prendere coscienza di loro stesse, a unirsi e diventare finalmente soggetto delle loro esistenze.
L'esperienza sessantottina aveva rappresentato uno stimolo che contribuì a porre al centro dell'agenda politica questioni come i rapporti tra i sessi, il diritto alla libertà sessuale e la struttura della famiglia, ma deluse le aspettative della componente femminile.
\\La lotta studentesca doveva, dal punto di vista delle ragazze, essere anche una sovversione dell'oppressione di genere e non solo di classe. Le donne che ne presero parte invece, in nome della liberazione, si ritrovarono sottoposte a obblighi comportamentali e relegate in una posizione subordinata rispetto ai loro compagni uomini: la struttura gerarchica e patriarcale non fu modificata.

\paragraph{}Un'analisi lucida di questa condizione venne pubblicata dal gruppo femminista Cerchio Spezzato composto da sole donne in un opuscolo indirizzato al genere femminile e distribuito all'Università di Trento nel 1971.
È un invito alla lotta.
\\A scriverlo furono ragazze che presero parte al movimento studentesco e che a esso avevano affidato anche la speranza della fine dell'oppressione dell'uomo sulla donna.
Le loro aspettative furono però deluse: \enquote{I gruppi di lavoro politici hanno riverificato la nostra sistematica subordinazione: noi siamo «la donna del tal compagno», quelle di cui non si conoscerà mai la voce, limitate al punto di arrivare a crederci realmente inferiori}\footcite{CerchioSpezzato}.
Tra i colpevoli di questa discriminazione fu individuato il capitalismo che \enquote{dopo aver sfruttato indiscriminatamente donne uomini e bambini (nella prima fase dell'industrializzazione) utilizzando il rapporto di dipendenza della donna rispetto all'uomo, l'ha espulsa dal processo produttivo ricacciandola nella famiglia. La donna è diventata sempre più schiava domestica, produttrice di lavoro domestico educatrice di bambini}\footcite{CerchioSpezzato}.
\\Il gruppo chiese alle donne di prendere coscienza della propria condizione, un'oppressione che trascende le classi sociali e nasce solo dall'essere di un sesso biologico diverso, e di unirsi.
Era giunta l'ora che anche le donne diventassero soggetti politici.
Il pamphlet si conclude con un'interessante analisi sul nuovo concetto di amore libero, per cui i giovani avevano lottato, affermando che la libertà sentimentale e sessuale delle donne continuava a essere subordinata a quella dei compagni uomini. Era questo un tipo di sessualità che confermava e rafforzava le strutture tipiche delle relazioni borghesi nelle quali la donna \enquote{non si pone come soggetto, ma è "l'altro"}\footcite{CerchioSpezzato}.
Il sesso femminile continuava così a determinarsi in relazione all'uomo, non come individuo autonomo.
C'era bisogno di un movimento solo femminile alla guida di una nuova forza politica per mettere in discussione i rapporti tra sessi per liberarsi.

\paragraph{}È nell'estate 1970 che Carla Lonzi, Elvira Banotti e Carla Accardi elaborarono quello che convenzionalmente è considerato l'atto di nascita del femminismo degli anni Settanta.
Il manifesto pubblicato sulla rivista \textit{Rivolta femminile} invoca a deculturalizzare e destrutturalizzare il sistema patriarcale per realizzare una tabula rasa sulla quale le donne, prive di condizionamenti, potessero riscrivere una politica inedita.

Il femminismo di questi anni si declinò in diverse direzioni, non fu un movimento unito e coerente.
Rimasero però sempre presenti alcuni nuclei tematici tra cui la profonda critica ad un sistema patriarcale teorizzato come un fatto naturale, l'uguaglianza tra uomo e donna, la lotta per il rispetto al corpo e l'autodeterminazione in ambito procreativo e sessuale.
\\Era forte la volontà di affermare la necessità di un intervento politico al suono dello slogan \textit{il personale è politico} e di indagare la base del dominio maschile all'interno della sfera sessuale.
Il sesso è un atto politico e di potere e le donne ne presero consapevolezza.
Iniziò così la \enquote{seconda rivoluzione sessuale}\footcite{Balestracci} con al centro una riflessione sulla sessualità che comportò un ripensamento delle concezioni nazionali della morale pubblica, di lecito e illecito, di privato e pubblico sia a livello legislativo sia culturale.


\subsubsection{Il nuovo interesse verso il corpo e la sessualità}
Il Novecento fu caratterizzato da studi e discussioni, anche scientifiche, sulla sessualità e sul piacere aprendo le porte non solo alla libertà delle donne nel rapporto eterosessuale, ma anche ad altre forme di sessualità, prima condannate, come la bisessualità, l'omosessualità e la masturbazione\footnote{la masturbazione nel Settecento venne considerata causa di malattie, nell'Ottocento fu sintomo di squilibri mentali, nel Novecento il nesso tra autoerotismo e malattie mentali fu scardinato, ma l'atteggiamento verso il sesso in tutte le sue forme, esclusa la procreazione, rimaneva conservatore e moralistico}.
\\Norbert Elias e Sigmund Freud rifletterono a lungo sulle conseguenze del progresso e della civilizzazione, un processo che ebbe conseguenze sociali e psichiche il cui principio regolatore fu il controllo delle pulsioni e degli istinti e quindi anche la rinuncia del totale soddisfacimento dei desideri sessuali.
Interessante riguardo il meccanismo di privatizzazione e il valore sociale della sfera sessuale e corporea è la riflessione di Norbert Elias che in \textit{La civiltà delle buone maniere}\footnote{edito in Italia solo nel 1982}, sviluppò una riflessione sull'automatismo psichico che a causa dell'aumento di civilizzazione portò a sopprimere la dimensione corporea di individuo: il sesso diventò un argomento tabù. 

Già Freud aveva affermato come la civiltà moderna si fosse edificata sulla repressione delle pulsioni; ogni individuo avrebbe infatti sacrificato una parte della sua libertà personale per garantire un'esistenza pacifica tra simili.
È il passaggio da stato di natura a contratto sociale.
\\La società non accetta una sessualità fine a se stessa: \enquote{la civiltà odierna intende permettere le relazioni sessuali solo sulla base di un legame unico e indissolubile tra uomo e donna, non accetta la sessualità come fonte di piacere fine a sé stessa, disposta a tollerarla solo come mezzo finora insostituito per la propagazione della specie}\footcite{Freud}.
Il controllo sulle proprie pulsioni si trasformò in autocostrizioni e automatismi mentali che influirono sulla libertà sessuale.
L'obiettivo era quello di far rientrare in schemi definiti gli istinti così che potesse essere fondata una società ordinata e armonica: per la conquista della sicurezza era necessaria la rinuncia alle pulsioni sessuali.

\paragraph{}Capire e spiegare il funzionamento del corpo è un passaggio fondamentale per scardinare i tabù e istruire la popolazione.
Erano necessari degli studi scientifici, psicologici e sociali perché le donne avessero gli strumenti per emanciparsi.
Essi consentirono di sollevare la sessualità dalla vergogna e dal pudore in cui era confinata.

Tra i primi a impegnarsi a livello scientifico nel cercare di spiegare la sessualità delle donne e le modalità dell'orgasmo fu Freud.
Egli maturò, nei suoi \textit{Tre saggi sulla sessualità} (1905), una teoria psicanalitica dedicata alla sessualità delle donne dalla loro infanzia fino alla maturità.
Ritenne che una tappa fondamentale nello sviluppo delle bambine è il momento in cui prendono coscienza di non avere il pene, la cosiddetta \enquote{invidia del pene}, che si tramuta in un desiderio di essere loro stesse dei maschi, sentono la mancanza di qualcosa.
Le persone di sesso femminile sarebbero quindi, per motivi biologici, incomplete.
\\Inoltre distinse l'orgasmo clitorideo da quello vaginale: il primo sarebbe appartenuto alle ragazze in età di sviluppo, con la maturazione sarebbero invece passate ad avere un orgasmo di tipo vaginale.
In questa transizione la donna avrebbe abbandonato la sua eccitabilità a favore di istinti procreativi.
Questa teoria sembra relegare la possibilità dell'orgasmo a un rapporto eterosessuale, a una dominazione maschile.
\\Studi di questo genere, oggi smentiti, confermarono e rafforzarono la subordinazione della donna all'uomo già radicato nella società.
La subalternità era una questione biologica e non culturale.

\paragraph{}La teoria sviluppata da Freud segnò l'inizio di un dibattito internazionale fondamentale per la liberazione del corpo femminile.

Ad attaccare la tesi di Freud fu Anna Koedt che nel suo saggio \textit{Il mito dell'orgasmo vaginale} (1968) negò l'esistenza di un orgasmo vaginale sostendo che la\footnote{Esiste un dibattito sul genere del sostantivo, in particolare il movimento femminista sostiene che l'utilizzo del femminile possa essere parte del processo di riappropriazione della propria sessualità, l'Accademia della crusca riconosce entrambi i generi come corretti. L'importanza del linguaggio per la lotta femminista diventa centrale negli anni '80 e '90 ed è tornato anche oggi} clitoride fosse il vero centro della sessualità e sottolineando come l'anatomia e gli studi scientifici più moderni confermino ciò.
Aggiunse che la frigidità di cui sono accusate le donne a lungo considerata un problema psicologico in realtà altro non è che il risultato di stimolazioni convenzionali favorevoli al raggiungimento dell'orgasmo maschile, senza interesse per la soddisfazione sessuale della donna.
Secondo un'inchiesta di \textit{Panorama} del 1978 una donna su due fingeva di raggiungere l'orgasmo perché l'uomo non sentisse la sua virilità sminuita\footcite{Balestracci4}.
Le donne sentivano ancora la necessità di subordinare il proprio corpo alle necessità maschili e anteporre il piacere dell'uomo a quello personale.
\\Secondo Koedt le donne \enquote{ sono state definite sessualmente nei termini che appagano gli uomini; la nostra biologia non è stata analizzata in modo appropriato. Invece, siamo state alimentate con il mito della donna liberata e dell'orgasmo vaginale, un orgasmo che di fatto non esiste}\footcite{Koedt}.
Ridefinire la sessualità femminile è necessario.

Nel suo studio prestò attenzione ai risvolti sociali dell'ignoranza legata al tema e alle false teorie ormai prese per certe.
Come si è detto, gli studi passati tendevano a mantenere l'ordine nel rapporto di subordinazione nella coppia, essendo il sesso, anche, un rapporto di potere, e portarono le stesse donne a considerare a lungo l'atto sessuale come un momento dedicato al piacere esclusivo dell'uomo.

A concordare con Koedt fu Carla Lonzi, la quale affermò che l'orgasmo vaginale non era per le donne il piacere più completo, ma \enquote{Il piacere ufficiale della cultura sessuale patriarcale. Raggiungerlo per la donna significa sentirsi realizzata nell’unico modello gratificante per lei: quello che appaga le aspettative dell’uomo}\footcite{Lonzi}.
Solo con l'abolizione di questo sistema e con la prese di coscienza della donna sulla propria sessualità può esistere il femminismo e la fine del patriarcato.

\paragraph{}In America Alfred Kinsley, biologo e sesuologo, aprì la strada, tra gli anni '40 e '50, agli studi sociali sul comportamento sessuale della popolazione basati su interviste e questionari.
Anche in Italia il cambiamento dei comportamenti delle ragazze e il loro nuovo rapporto con la sessualità diventarono argomenti di dibattito pubblico.

Gabriella Parca\footnote{Parca fu una giornalista, fondatrice del mensile \textit{Effe} nel 1972 primo rotocalco italiano di controinformazione al femminile e nel 1975 di uno dei primi consultori laici italiani} fu autrice delle prime inchieste sui rapporti tra sessi, già alla fine degli anni '50 pubblicò \textit{Le italiane si confessano} dal quale fu tratto anche un film..
Il volume raccoglie alcune della lettere che l'autrice aveva ricevuto e delinea una società italiana ancora profondamente maschilista, nella quale molto è ancora taciuto.
Il suo lavoro anticipò di pochi anni l'inchiesta \textit{Comizi d'amore} di Pasolini.

Ad anticipare il discorso pubblico e la politicizzazione della liberalizzazione sessuale furono gli studenti del Liceo Parini di Milano pubblicando l'articolo \textit{Che cosa pensano le ragazze d’oggi} nel 1966 sul loro giornale \textit{La zanzara}.
Al centro del sondaggio ci sono il divorzio, la contraccezione, di cui era vietato discutere pubblicamente, l'assenza di educazione sessuale: fu uno scandalo nazionale.
\\Le ragazze della nuova generazione vivevano più liberamente la sessualità, senza sensi di colpa morali conseguenti alle idee propagate dall'etica cattolica, volevano rapporti prematrimoniali grazie all'uso di contraccettivi e un futuro lavorativo non all'interno delle mura domestiche: \enquote{Non vogliamo più un controllo dello stato e dalla società sui problemi del singolo e vogliamo che ognuno sia libero di fare ciò che vuole, a patto che ciò non leda la libertà altrui. Per cui, assoluta libertà sessuale e modifica totale della mentalità”} \footcite{Zanzara}.
I giovani iniziarono a farsi portavoce dei discorsi riguardanti la sessualità.
L'emersione dal silenzio delle questioni relative al corpo e alla sessualità provocarono scalpore in una società nella quale questi argomenti erano sempre stati censurati.

Durante gli anni '70 i giornali si riempirono di inchieste sul sesso: alle donne venivano chieste opinioni sulla verginità, sul piacere femminile, sui rapporti prematrimoniali ed extramatrimoniali.
Sembra di vivere in un interrotto \textit{Comizi d'amore}, un continuo studio sulla società.
Le italiane intervistate sembravano essersi emancipate dai tabù relativi alla verginità o dal divieto di non poter avere rapporti prima delle nozze.
Dai sondaggi pubblicati l'impressione che se ne ricava è di donne che hanno il controllo delle proprie azioni e del proprio corpo.
\\La realtà sociale era però variegata, tradizione e ignoranza convivevano. 
Da diversi studi e sondaggi emersero alcune contraddizioni forse dovute a una trasformazione dei comportamenti a cui non corrispose un mutamento altrettanto veloce e radicale della cultura e dell'educazione.
\\La libertà con cui si iniziava a parlare di questi temi non sembrò abbattere davvero né i tabù né la disinformazione su contraccezione, concepimento e funzionamento del corpo femminile\footnote{solo nel 1985 il servizio di assistenza telefonica creato dall'Associazione Italiana per l'Educazione Demografica, nata nel 1953 per iniziativa di alcuni circoli intellettuali e politici di area socialista e radicale per favorire il controllo della nascita e una cultura consapevole della sessualità, ricevette 12mila chiamate da uomini e donne di ogni età con dubbi relativi alla sfera sessuale}.

\paragraph{}Gli studi scientifici e antropologici furono necessari perché venisse eliminata la credenza che la donna fosse inferiore all'uomo per natura e si diffondesse invece l'idea che questa condizione fosse il risultato di studi non verificati, di stereotipi e di tradizioni antiche e maschiliste.
Fu il corpo una degli elementi chiave per definire la nuova identità politica delle donne.
\\L'enfasi sull'orgasmo e sul corpo femminile ebbe un ruolo importante nell'educare le donne permettendo loro di prendere consapevolezza della loro condizione di subordinazione e liberarsene, ma produsse anche distorsioni.
La continua attenzione generò nella società un fenomeno di ipersessualizzazione  del corpo femminile.


\subsubsection{Il corpo sessualizzato}
In anni in cui sembrò finalmente iniziare il desiderato processo di liberalizzazione si delinearono forti contraddizioni.
La più grande e drammatica, che arriva fino a oggi, fu la commercializzazione, sessualizzazione e  spettacolarizzazione del corpo femminile.
Il sesso, o almeno alcuni aspetti di esso, non fu più relegato in una sfera privata, quasi scandalosa, e a pagarne le conseguenze fu la donna.
Il suo corpo divenne merce, un'immagine prodotta in funzione, ancora una volta, del soddisfacimento dello sguardo maschile.
\\La donna torna a essere oggetto e non soggetto agente che si autodetermina.

Il corpo femminile è sempre stato nel corso della storia oggetto di rappresentazioni che dovevano favorire e giustificare la posizione subordinata a cui erano relegate le donne.
È un corpo esibito e utilizzato dalle ideologie, ma non auto-rappresentato.
Dalla Rivoluzione Francese e lungo l'Ottocento la corporeità femminile divenne centrale nel racconto della nazione, venne usato come allegoria della nazione.
Sono immagini di donne spesso vestite con lunghe tuniche che lasciano parte del corpo scoperto e circondate di simboli patriottici\footnote{si può pensare al dipinto \textit{La libertà che guida il popolo} (1830) di Eugène Delacroix,, Museo del Louvre, Parigi, che rappresentò la Libertà conquistata dalla Francia}.
\\Le figure femminili utilizzate sono spesso cariche di erotismo, sono soggetti desiderabili che si offrono a guida morale, non politica, del popolo.
Sono l'emblema di una patria materna che nutre e si prende cura dei suoi figli, del suo popolo.

Nel Novecento con l'avvento dei nazionalismi il ruolo femminile esaltato dalla propaganda era quello della donna procreatrice.
Il corpo delle donne era strumento dello Stato, inserito in un rigido programma con finalità demografiche\footnote{durante il fascismo vennero vietate la vendita e la promozione dei contraccettivi, anche la discussione di essi fu proibita, l'aborto divenne reato contro la patria}.
Il bene della collettività e della discendenza era ritenuto superiore all'autodeterminazione e la donna doveva sottomettersi a esso.

\paragraph{}I media della società consumistica e capitalistica cambiarono totalmente il modo di rappresentare la donna.
La diffusione dei beni di consumo degli anni Cinquanta creò la figura della donna casalinga, una casalinga sensuale.
È un immaginario lontano dalla realtà dei servizi di casa.
Questa è la donna consumatrice, quella che deve comprare i prodotti che vede in televisione.
\\Il linguaggio pubblicitario si fece carico di propagare questo nuovo modello in cui la donna trovava la sua gratificazione nel lavoro domestico e nel presentarsi bella e sempre curata davanti al marito.
Si realizzava nell'essere moglie e madre e nel compiacere il proprio marito; questo era il suo lavoro.
 
\paragraph{}Accanto a figure di donne casalinghe perfette si diffusero nuovi modelli femminili nei cinema e nei programmi tv.
Se la nudità non era ammessa in Rai, le allusioni di dimensione erotica non erano  del tutto assenti\footnote{nell'intervista del 31 ottobre 1959 all'attrice americana Jayne Mansfield che per l'occasione non indossò gli abiti scollati tipici del suo personaggio l'intervistatore Mario Riva non evitò di fare apprezzamenti sul suo corpo}. 
Nel mondo televisivo iniziava così a prendere forma l'idea del corpo femminile come oggetto dello sguardo maschile.
\\Se da una parte l'esibizione del corpo era non solo tollerato ma cercato dagli italiani, dall'altra il 29 novembre 1956 durante la messa in onda del varietà \textit{La piazzetta} la visione delle gambe di Alba Arnova, che indossando calze color carne sembravano nude, in prima serata sulla Rai destò grande clamore.
La colpa attribuita ad Arnova fu di aver voluto dare l'impressione di una intenzionale nudità del suo corpo.
È il primo scandalo che coinvolse la Rai, il programma venne sospesa e Arnova allontanata dal mondo televisivo.
La parte più conservatrice della società si scandalizzò di nuovo nel 1970 quando Raffaella Carrà esibì per prima l'ombelico in diretta televisiva.
Il corpo femminile quindi poteva essere mostrato, esibito e commentato, ma doveva essere una decisione di un uomo.

La società dei consumi fece in modo che i media iniziassero a diffondere immagini femminili erotizzate: iniziava quella che può essere definita liberalizzazione della pornografia.

\paragraph{}La sessualizzazione riguardò ogni aspetto della vita quotidiana.
Gli italiani si ritrovarono circondati da pubblicità sessualmente allusive, da giornali rivolti al pubblico maschile e produzioni cinematografiche di commedie erotiche proiettate in sale a luci rosse.
Si moltiplicò la diffusione di prodotti artistici che diedero voce ai temi del corpo e dell'eros opponendosi ai tabù dei decenni precedenti.

Il dilagare dell'erotizzazione nei lavori letterari e cinematografici fu argomento di diverse inchieste che coinvolsero i personaggi pubblici dell'epoca.
Tra questi Fortini che considera l'erotismo \enquote{il più vulgato e accessbile dei tabù}\footcite{fortini}.
L'apparente tolleranza e attenzione alla questione sessuale nasconde un esercizio di potere più profondo e stratificato.
\\L'erotismo non può essere letto in chiave rivoluzionaria poiché esso sviluppandosi non si associò a un serio mutamento politico, ma solo a dinamiche economiche e commerciali.

Il 1969 vide aprire in Veneto il primo sexy shop, sei anni dopo una nuova legge stabilì la non punibilità di rivenditori ed editori di materiale pornografico\footnote{purché non venissero mostrate parti intime di minori di 16 anni}.
Solo un anno prima il direttore della rivista erotica \textit{Kent} era stato condannato a tre mesi di reclusione per la diffusione di materiale osceno.
L'intellettuale Luciano Bianciardi prese posizione nella vicenda schierandosi a favore della libertà di stampa e della depenalizzazione del pornografico.
La fruizione pornografica passò dalla clandestinità alla facile fruizione in edicole, cinema e negozi.
La fine degli anni Sessanta è un momento decisivo per la storia della diffusione del porno che da piccolo mercato di nicchia e illegale velocemente si trasformò in prodotto di massa. 
Gli italiani avevano accesso al mondo pornografico senza avere gli strumenti per goderne in modo responsabile.
\\I confini della legalità di contenuti considerati osceni rimasero confusi e la determinabilità del reato era affidata alla sensibilità del giudice.
In campo letterario un momento decisivo fu il processo per oscenità che subì Aldo Busi nel 1990 a causa di scene di sesso omosessuale tra uomini in \textit{Sodomie in corpo 11}.
Venne assolto con formula piena e questo fu l'ultimo processo contro la pornografia in opere letterarie.
Solo nel 2016 l'oscenità venne depenalizzata.

Credere di abbattere i tabù presenti nella società italiana da secoli attraverso la pornografia è utopico, non libera davvero i corpi.
L'erotismo così come si è presentato in Italia non fu espressione di libertà, ma volontà del potere, è uno spettacolo voluto dalle forze economiche.
\\Il cinema diventò il luogo di un'alienazione dovuta alla sovraesposizione mediatica del sesso che depotenzia il vero desiderio erotico e la possibilità di sviluppare un discorso sulle vere possibilità liberatorie della sessualità.
Soprattutto nelle zone più periferiche le trasformazioni antropologiche e dei costumi sessuali travolsero la cultura tradizionale senza confrontarsi con essa.
Le sale a luci rosse sostituirono i cinema ed è in questi luoghi che i giovani cercano di divertirsi la sera o nei weekend, senza però riuscirci.
La società sessualmente repressa, soprattutto quella provinciale, venne investita dalla promozione del consumo sessuale che traumatizzò gli italiani.
\\Nel giro di davvero troppo poco tempo l'Italia dall'essere contro ogni forma di rappresentazione esplicita a diventare uno dei paesi più pornografizzati d'Occidente.
Non c'è più separazione tra vita sessuale e sfera pubblica.

\paragraph{}Considerando la produzione cinematografica non erotica e la rappresentazione di personaggi femminili sul grande schermo è interessante il saggio \textit{Visual Pleasure and Narrative Cinema} (1975) di Laura Mulvey.
La critica indagò la costruzione dell'immagine della donna utilizzando la psicanalisi per comprendere in che modo il cinema rivelasse la differenza sessuale.
Mulvey affermò che l'esperienza cinematografica fosse progettata in funzione della soddisfazione del desiderio del maschio bianco ed eterosessuale che con il suo sguardo sentiva di possedere la donna il cui ruolo era puramente erotico e si esauriva nel soddisfare il desiderio maschile.
Il personaggio femminile era rappresentato come un accessorio, la sua ombra e la sua presenza era giustificata dalla presenza dell'uomo, dalla sua esistenza in relazione a lui.

In tempi recenti la rappresentazione sempre cambiata, ci sono sempre più film con donne protagoniste, sull'onda della necessità di inclusione, ma con sotto uno sguardo più critico emerge che nella produzione cinematografica non è ancora accettato dal pubblico una rappresentazione paritaria tra i sessi.
I film pensati per un pubblico maschile ancora oggi presentano i personaggi femminili come ornamento dell'uomo.
Nel cinema le donne, oltre a essere pagate il 25\% in meno, sono sottorappresentate, hanno ruoli di contorno e pronunciano in media meno battute rispetto ai loro colleghi.
Hanno però il doppio delle possibilità di dover recitare scene di nudo\footnote{ricerca commissionata da New York Film Academy: www.nyfa.edu/film-school-blog/gender-inequality-in-film-infographic-updated-in-2028}.
Le figure femminili nei film servono ancora oggi a completare il personaggio maschile e compiacere lo sguardo dello spettatore.

\paragraph{}La trasformazione della sessualità in un bene di consumo, sottoposto alle logiche di mercato, fu il risultato del nuovo modello economico che neanche la morale cattolica e conservatrice riuscì ad arrestare.
La una società consumistica interessata alle vendite e al successo favorì e promosse lo sfruttamento e la commercializzazione della libertà sessuale che le donne cercavano di ottenere.
La donna si tolse i panni di procreatrice e di casalinga per mettere quelli di icona sexy.
Il sesso dall'essere censurato fu conformato e integrato in un sistema di commercializzazione, la donna da oggetto per il controllo demografico passando per la liberalizzazione divenne oggetto per la soddisfazione dello sguardo dell'uomo

Il bombardamento da parte dei nuovi mezzi di comunicazione sul tema della sessualità, la produzione di un immaginario di donna ipersessualizzata si inseriscono in un processo di pseudoliberazione che tende, con il passare del tempo, ad attribuire alla donna un nuovo ruolo oggettivato e stereotipato invece che liberarla.

La rivoluzione sessuale va quindi letta con una duplice connotazione: da una parte non può che essere guardata in modo positivo come portatrice di valori culturali moderni, dello smascheramento della condizione subordinata femminile e di grandi conquiste a livello legislativo, ma dall'altra afferma un nuovo set di regole che pur avendo le sembianze di libertà in realtà rinchiudono la donna e il suo corpo in un meccanismo di subordinazione alle leggi di mercato.
