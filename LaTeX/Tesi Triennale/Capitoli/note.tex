Il titolo, che fa pensare all'improbabilità di una situazione dal momento che è impossibile che i porci abbiano le ali, è spiegato dall'autrice in questo modo: \enquote{le ali sono sicuramente nel comunque decidere di vivere la sessualità in ogni caso in modo problematico e politico cercando di capire di andare avanti}\footnote{dall'intervista nel programma \textit{Match} condotto da Arbasino \url{ https://www.raiplay.it/video/2016/11/Susanna-Agnelli-e-Lidia-Ravera-c840bd18-9af6-4cf0-82c7-a7cb5907955d.html}}.
È un titolo, abbinato alla grafica della copertina era in grado di suscitare interesse.





A differenza di quella precedente, la rivoluzione degli anni Settanta fu caratterizzata da studi e discussione scientifiche sulla sessualità e sul piacere\footnote{approfondimento nel cap.2.4}, aprendo le porte non solo alla libertà delle donne nel rapporto eterosessuale, ma anche ad altre forme di sessualità, prima condannate come la bisessualità, l'omosessualità e la masturbazione\footnote{la masturbazione dal Settecento venne considerata causa di malattie, dall'Ottocento fu sintomo di squilibri mentali, nel Novecento il nesso tra autoerotismo e malattie mentali fu scardinato, ma l'atteggiamento verso il sesso in tutte le sue forme, esclusa la procreazione, rimaneva conservatore e moralistico}.


Ma è dagli anni '70 che film, pubblicità, riviste illustrate e dibattiti popolarizzarono il discorso sulla sessualità come mai prima.
L'intera società ne era coinvolta, dalla famiglia la cui struttura tradizionale era in crisi, alla classe politica fino al Vaticano, sembravano lontani i tempi dello scandalo di \textit{La Zanzara}.




DISCORSO SU STORICISMO


VOLEVO I pANTALONI
Al centro della riflessione pubblica c'erano ancora delle pratiche sociali che andavano discusse tra cui quella del \enquote{codice dell'onore}; fu un evento di cronaca del 1965 a scuotere la coscienza collettiva, il caso di Franca Viola.
La ragazza, appartenente a una modesta famiglia di Alcamo, venne rapita all'età di 17 anni dall'ex fidanzato Filippo Melodia, nipote di un mafioso locale, e violentata, la famiglia Melodia propose il tradizionale matrimonio riparatore che avrebbe salvato il ragazzo dal carcere e l'onorabilità della ragazza ma il padre di lei non acconsentì e collaborò con la polizia per la liberazione della figlia, lei stessa una volta libera rifiutò di sposarsi; fu la prima donna a rifiutare pubblicamente il matrimonio riparatore e aprì un processo che si concluse con una condanna di 11 anni per Filippo Melodia.
Il Codice Penale italiano prevedeva il matrimonio riparatore secondo il quale il colpevole di stupro sarebbe stato assolto dal reato se avesse sposato la persona violentata, era concepito come una forma di risarcimento verso la donna che secondo una concezione patriarcale, avendo perduto l'onore, non sarebbe più potuta essere presa in moglie da nessun altro uomo, e il delitto d'onore, lascito del Codice Zanardelli (1889)\footnote{che prevede attenuanti al delitto sia se commesso da parenti maschi sia femmine della vittima} e del Codice Rocco\footnote{le attenuanti sono riconosciute solo al marito, padre e fratello della donna uccisa}, che prevedeva pene minori per un omicidio commesso per salvaguardare l'onore del proprio nome o della propria famiglia.
Solo nell'agosto 1981 il Parlamento approvò l'abrogazione della rilevanza penale della causa d'onore e solo dal 1996 la violenza sessuale divenne reato contro la persona e non contro la morale.

stupro è espressione di potere, non desiderio


La verginità da fine Ottocento era invocata come elemento fondamentale perché la donna fosse considerata d'onore, era una costruzione culturale che manifestava il desiderio di controllo della sessualità e del corpo femminile.

Appunti brugnolo su verga (siciliani) e su deledda canne al vento
vero personaggio riv è zia, protagonista torna alla tradizione
opera racconta un fallimento che riguarda tutti

Famiglia:\paragraph{} L'Italia stava cambiando, l'economia si era rivoluzionata, la tradizione culturale cercava di modernizzarsi.
A rimanere antico era il sistema legislativo.
Il codice di diritti e doveri dei coniugi rimaneva invariato sullo stampo del Codice Pisanelli (1865) che conservando il marito a capo della famiglia, la patria potestà, l'indissolubilità del matrimonio ed eliminando solo la necessità dell'autorizzazione maritale per ogni transizione economica, manteneva profondamente radicata la famiglia patriarcale nella quale l'uomo aveva il controllo e poteva mantenerlo anche facendo uso della violenza.
A livello penale rimase in vigore il Codice Rocco (1930), codice fascista che considerava la violenza carnale un reato morale e la contraccezione e l'aborto reati contro la stirpe.
Si creò così uno scenario carico di contraddizioni non risolte.
La nuova Italia che aveva aperto alle donne la possibilità di votare doveva fare i conti con il passato.

Dal Sessantotto in poi molte furono le battaglie per modificare la struttura della famiglia\footnote{Ginsborg considera fondamentale lo studio dell'istituzione familiare per rileggere la storia nazionale italiana} e togliere la donna dall'oppressione esercitata dal marito.
Si aprì una stagione di grande fermento e di lunghe battaglie per tutelare l'autodeterminazione e perché venisse attuata una riforma del diritto di famiglia non più adatto al modo di pensare e vivere la famiglia, le relazioni di coppia e la sessualità.

Le grandi trasformazioni che avevano investito l'Italia, tra cui il boom economico, la decrescita dell'agricoltura, l'emigrazione, mandarono in crisi l'organizzazione gerarchica e autoritaria della famiglia.
Si capovolse il rapporto tra sesso femminile, famiglia e società, venne meno la netta separazione tra l'uomo \enquote{breadwinner}, produttore del reddito familiare e la donna dedita al focolare domestico.
Al suo posto iniziò ad affermarsi il nuovo modello di dual-breadwinner che aprì la strada, ancora lunga, alla parità di genere in ambito economico-lavorativo che ambiva a superare la tensione tra cura e occupazione e a favorire la libertà di scelta.

Il 1970 fu l'anno in cui sembrò concludersi la lunga lotta\footnote{La prima proposta di legge risaliva al 1965 e fu avanzata dal socialista Loris Fortuna che propose un testo moderato che limitava questo diritto ad alcune situazioni definite ma la Dc bloccò l'iter parlamentare, nel 1969 vennero fatte alcune modifiche, il Pci, dopo lunghe discussioni ed essere giunto alla conclusione di essere favorevole al divorzio e al riconoscimento dei mutamenti avvenuti nella società ma di voler anche difendere la famiglia la cui funzione di stabilizzatrice sociale non doveva essere messa in discussione, assicurava il suo appoggio mentre la Chiesa invitava i suoi fedeli a pregare per allontanare questa possibilità} per introdurre il divorzio, era uno storico obiettivo del femminismo.
La legge per il divorzio venne riconfermata dalla vittoria del no al referendum per l'abrogazione nel 1975\footnote{vinse il no con il 59\% dei voti \url{https://www.istat.it/it/files//2019/03/cap_9.pdf}}

Da questo momento in poi il movimento femminista si dedicò alla battaglia a favore dell'aborto. L'interruzione volontaria di gravidanza pur essendo illegale era un fenomeno diffuso, secondo uno studio del 1961 firmato dalla giornalista Milla Pastorino\footcite{Pastorino} ogni 100 gravidanze portate a termine 50 erano interrotte.
Era necessario che la questione dell'aborto diventasse pubblica.

Il movimento  si dedicò non solo a diffondere il principio di autodeterminazione, a rilanciare il tema della contraccezione e predisporre a livello nazionale iniziative per favorire una consapevole gestione della propria sessualità, ma si impegnò nell'offrire assistenza alle donne che avevano bisogno organizzando viaggi verso le città estere dove l'aborto era regolamentato, a creare centri dove effettuare visite ginecologiche effettuate da medici volontari.
Alcuni gruppi decisero inoltre di fondare anche nuclei di autogestione dell'aborto dopo aver imparato il metodo dell'aspirazione agendo clandestinamente e assumendosi la responsabilità clinica e penale.

Nel maggio 1978 vennero approvate le \enquote{Norme per la tutela sociale della maternità e sull'interruzione volontaria di gravidanza}, poi confermata da un referendum popolare nel 1981.
Questa legge fu una grande conquista, l'aborto non era più reato, ma lasciò una parte del movimento femminista con l'amaro in bocca poiché l'autodeterminazione delle donne non venne tutelata: prima di abortire era d'obbligo consultarsi con un medico e un assistente sociale e poi aspettare una settimana di \enquote{meditazione} prima di poter essere sottoposte all'intervento, le ragazze sotto la maggiore età dovevano avere il permesso dei genitori e infine veniva riconosciuto ai medici il diritto all'obiezione di coscienza.

Il diritto di famiglia, ancora regolamentato dalla codificazione del 1942 ispirata al modello napoleonico, vedeva le relazioni in modo gerarchico.
La legislazione doveva cercare di riequilibrare le relazioni all'interno del nucleo familiare.
Nel 1975 la riforma investì un ampio spettro di questioni come il matrimonio, la filiazione, la separazione.
Dopo anni di lotte per la parità venne affermata l'uguaglianza tra i coniugi che sposandosi assumevano gli stessi diritti e doveri e abolita la figura del capofamiglia.
La rivoluzione sessuale introdusse nelle relazioni di coppia una novità: in esse doveva essere presente l'intesa affettiva, romantica ed erotica e non doveva essere vissuta come un lavoro, da cui lo slogan \enquote{Il matrimonio non è una carriera!}.
È un'unione basata sui valori del consenso e delle parità.





PORCI
inchieste su panorama verginità e fingere per uomo
footnote{Secondo un'inchiesta del 1969 pubblicata da \textit{Novella 2000} sul comportamento sessuale delle donne la metà delle intervistate aveva avuto rapporti prematrimoniali e un terzo extramatrmoniali.\footcite{Novella}
In un'altra del 1977 pubblicata da \textit{Panorama} in collaborazione con alcuni sociologi risultò che per l'85\% delle intervistate la verginità non era più un tabù se riferito a se stesse, ma rimase radicato se proiettato sulle figlie\footcite{Balestracci3}.
La verginità da fine Ottocento era invocata come elemento fondamentale perché la donna fosse considerata d'onore, era una costruzione culturale che manifestava il desiderio di controllo della sessualità e del corpo femminile.}

comunità vs individualità

gattopardismo: Antonia si comporta come una donna libera, ma  è apparente, vive le stesse paure di generazioni prima
Cerchio spezzato?







BOH
Un momento di svolta si ebbe però con la Resistenza che vide una partecipazione attiva delle donne.
La maggior parte ricoprì un ruolo strategico, di staffetta o di assistenza, ma ci fu anche chi prese parte alle azioni armate.
Alla fine dell'azione partigiana si affermò la tendenza a rimuovere dalla memoria collettiva la presenza femminile che aveva oltrepassato i tradizionali limiti dei ruoli maschili e femminili.
L'allontanamento della figura della donna nella Resistenza andò di pari passo con l'esaltazione del combattente uomo, ma l'esperienza attiva antifascista fu determinante nella presa di coscienza femminile della necessità di un nuovo ruolo politico.

Da2Pier vittorio tondelli file 
Nel libro di TONDELLI si sente un forte legame con quegli anni12, ma non perché caratte-
rizzato da un discorso ideologico. Come lo stesso autore afferma, infatti, le sue opere “di-
mostrano sostanzialmente un rifiuto dell’ideologia”13, la cui assenza, però, non è concepita
come una mancanza: “Se ne occupavano gli altri [della politica, dell’ideologia]. [...] Questo
non vuol dire che sia uno scrittore disimpegnato, perché credo ci sia un importante risvolto
sociale nei miei libri.”14 Ed è proprio in questo “risvolto sociale” che si nasconde il rapporto
di Altri libertini con il decennio precedente, perché gli “episodi” del libro raccontano la re-
altà della generazione giovanile degli ultimi anni Settanta: giovani stanchi di politica e di
qualsiasi ideologia, scrutati nella loro vita privata e quotidiana, attraverso la quale, però, è
possibile intravedere anche una loro immagine collettiva (tanto che secondo certi critici in
alcune “storie di Altri libertini l’io è sostituito da un indefinito e corale noi”15). Infatti, tramite
i riferimenti culturali generazionali [i]l libro si offre come attendibile catalogo di tutti i miti e le figure dell’immaginario
giovanile di quegli anni, almeno relativamente ad un’area diciamo “alternativa” (movi-
mento del ’77 e dintorni, comprese le frange meno politicizzate) […].16


l’opera d’esordio
di PIER VITTORIO TONDELLI, Altri libertini (1980), in cui una certa riduzione è visibile, prima di tut-
to, nella scelta di un linguaggio fortemente contaminato da espressioni volgari e contenen-
te vari elementi fino ad allora poco consueti (se non del tutto sconosciuti) nell’ambito della
produzione letteraria “ufficiale”, ma altresì nella scelta delle vicende narrate che sono, allo
stesso modo, il riflesso di un certo “abbassamento” di stile.




CONFRONTO
