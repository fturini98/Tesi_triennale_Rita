\section{Il corpo, la politica e la letteratura}
\subsection{Da un dibattito sull'erotismo al cannibalismo}
Dagli anni '60 in poi il corpo e la sessualità divennero il tramite per parlare dell'intero contesto culturale italiano.
Si sviluppò un discorso sul valore centrale della corporeità nell'interpretazione della situazione storica-politica.
\\Nudità, sessualità, perversioni, mescolanze di generi e pornografia sono i modi in cui il corpo cerca di esprimersi perché, come notò Bazzocchi: "I corpi parlano, e il loro discorso non può più essere ignorato"\footcite{Bazzocchi}.

\paragraph{}La produzione letteraria del secondo Novecento cercò di rinvestire il corpo del suo linguaggio, di farlo esprimere.
La forza del corporeo e della sessualità divennero una costante nella cultura letteraria, soprattutto nella dialettica tra corpo grottesco e corpo perfettamente dato.
\\La distinzione tra i due fu teorizzata da Bachtin in \textit{L'opera di Rabelais e la cultura popolare. Riso, carnevale e festa nella tradizione medievale e rinascimentale} (1965).
Egli identificò il corpo grottesco come un corpo in divenire, in continua trasformazione e in grado di creare a sua volta altri corpi e delineò il nuovo canone corporeo come un corpo perfetto, delimitato, chiuso senza diramazioni.
Il corpo rigorosamente formato sembra essere quello preso come modello per la rappresentazione erotica, ma non si può non notare come il corpo "brutto" o deforme rivendichi in continuazione un suo spazio nella letteratura fino a diventare l'immaginario dominante negli anni '90 con il fenomeno del pulp e del cannibalismo.
Facendo riferimento, per ora, ai decenni precedenti si può pensare al corpo di Liliana nel \textit{Pasticciaccio}\footnote{ci si sta riferendo all'opera \textit{Quer pasticciaccio brutto de via Merulana} pubblicata in puntate nel 1946 sulla rivista \textit{Letteratura} e poi in volume nel 1957} che unisce nella sua immagine la connotazione erotica e quella violenta della morte.
Pasolini con la rappresentazione del corpo popolare rese complementari l'arcaicità del grottesco e l'immaginario del corpo erotico "bello".

\paragraph{} Si aprì un dibattito\footnote{partendo dall'inchiesta  \textit{Sesso e letteratura} a cura di Luigi Capelli su \textit{Corriere Lombardo} e l'inchiesta \textit{Otto domande sull'erotismo in letteratura} di \textit{Nuovi Argomenti} entrambe del 1961 
e i conseguenti interventi dei singoli intellettuali} sul rapporto tra letteratura, comunicazione e potere che coinvolse i maggiori intellettuali del secondo Novecento.
\\Autori come Fortini e Carlo Bo criticarono l'atteggiamento ottimista che concepisce l'eros come arma di liberazione e di rottura e videro nel nuovo clima di libertà il rischio che la produzione si commercializzi e trasformi in pornografia a basso costo.
Calvino affermò con decisione l'impossibilità di utilizzare l'erotismo come strumento di critica della società.
Egli interpretò, dopo l'operazione di ipersessualizzazione svolta dai media, l'erotismo come una forma di compensazione da parte di una società desessualizzata che ormai viveva la propria sessualità nei cinema a luci rosse, in un mondo falso.
Il riso e la rappresentazione di rapporti non antropomorfi divennero i suoi strumenti privilegiati per creare un discorso letterario sulla sessualità. 
\\Di diversa opinione fu Moravia che guardò con entusiasmo il processo di liberalizzazione della sessualità dai tabù che permise a chi scriveva di \enquote{rappresentare direttamente, esplicitamente, realisticamente e poeticamente in un'opera letteraria il fatto sessuale ogni volta che l'opera stessa lo renda necessario}\footcite{Moravia}.
Interrogato sulla sua concezione dell'erotismo l'autore spiegò che esso poteva essere considerato un ponte che l'individuo getta per collegare il mondo e se stesso, ma allo stesso tempo può anche essere un elemento di tensione e di distruzione della realtà.
A condensare in sé questa riflessione è Dino, il protagonista de \textit{La noia} (1960) che nella sessualità cerca gli strumenti per ritrovare un punto di contatto con la realtà, ma alla fine il corpo che tanto cerca di dipingere per appropriarsene lo allontana sempre più dal mondo reale.
Nel corpo egli vede la rappresentazione della realtà stessa nella sua completezza ed è per questo che non riesce a dipingerlo, non riesce ad afferrare l'assolutezza di esso.
Moravia utilizza la sessualità per approfondire un discorso filosofico, in un'intervista, spiega il processo in questo modo: \enquote{Nella \textit{Noia}, la crisi si situa innanzitutto sul piano filosofico come una crisi tra soggetto e oggetto. A un secondo livello, la crisi riguarda il rapporto tra artista e materia. E al livello più esplicito, il rapporto tra amante a amata. In tutti i tre i casi si tratta di una crisi di rapporto con il reale}\footcite{Moravia2}

Mentre Pasolini, come si è visto, ritenne di poter rappresentare solo il mondo del sottoproletariato presto lo sguardo degli autori si spostò verso quello borghese.
\\La borghesia, centrale nei nuovi fenomeni della cultura di massa, divenne soggetto di interesse culturale, saggistico e narrativo.
Dino, citato prima, personaggio di Moravia appartiene a questa classe sociale, anche il protagonista del breve romanzo \textit{Agostino} è di estrazione borghese e la famiglia risulta essere la ragione del turbamento sessuale.
Entrambi sono afflitti dell'incapacità borghese di avere contatti con la realtà e di vivere la sessualità al di fuori del controllo esercitato dalla famiglia.
L'erotismo ma anche i rapporti sentimentali nella vita borghese acquisirono un tono freddo e meccanico, automatizzato.

\paragraph{}Negli anni Sessanta l'interesse si spostò sulla dicotomia tra una sessualità, che si potrebbe dire, tradizionale e tipica della periferia e quella dei centri urbani stravolti dalla rivoluzione economica.
\\Luciano Bianciardi ne \textit{L'integrazione} (1960) si interessa a questo aspetto  e lo fa scrivendo di due fratelli che dalla periferia si trasferiscono in un centro e cercano, con esiti diversi, di integrarsi in un mondo diverse da quello che conoscevano.
Il sesso è sempre più connesso ai ritmi incessanti della produzione, al denaro.
Lo si vede nella descrizione del bordello di Chiaravalle in Bianciardi dove le prostitute sono presentate come merce, si muovono in modo meccanico e il focus è sul \textit{registrone}, il simbolo del consumismo.
Era presentato in questo modo: \enquote{Dalla porta giù in fondo ogni tanto irrompevano a branco le donne mezze nude, e facevano il giro per attecchire i clienti. Ognuna aveva la sua mossa: quella che sculettava, una per esempio aveva una ciocca di capelli tinta di bianco; un'altra agitava la lingua fuori dalla bocca. A tratti uno si alzava, con il viso duro e cattivo, la ragazza lo guidava all'ascensore. Ma prima si passava davanti a un tavolo, dove sedeva il contabile, con il suo registrone a partita doppia. La ragazza diceva il suo numero (...), il ragioniere lo ripeteva a riscontro e lo marcava sul registro. Siccome si pagava dopo, alla fine c'era da ripassare davanti al tavolo, per far registrare la somma in entrata, e il numero della ragazza in uscita. Ogni numero nella sua colonnina: una contabilità abbastanza complessa}\footcite{Bianciardi}.
Fino al XIX secolo la prostituta era venditrice e merce, ora è solo merce\footnote{è questa una riflessione di Walter Benjamin  nel saggio \textit{Parigi capitale del XIX secolo} in riferimento alla situazione parigiana, ma è una riflessione che si adatta anche al contesto italiano}.

I bordelli diventano nella sua opera metonimia dell'intera vita sociale della metropoli, una vita di incomunicabilità che aliena, spersonalizza le cui conseguenze sono l'uomo massa, frastornato davanti al supermercato della prostituzione, la prostituta moderna, parte di una catena di montaggio del piacere sessuale.
È significativa la riflessione dell'autore sulla differenza tra la prostituta moderna, merce esposta come in un supermercato del sesso, e l'antica  e nobile professione della \enquote{\textit{puttana}}.
\\La sessualità viene automatizzata e contabilizzata.
La vita erotica nella metropoli ha orari fissi, i bordelli sono fabbriche del piacere, i tempi sono controllati.
Anche il tempo libero è razionalizzato: non esiste più il passeggiare tipico delle relazioni del passato, è ora considerato una perdita di tempo. 
Il principio borghese del guadagno investe anche il sesso, anch'esso è sottoposto a imperativi produttivi e di ottimizzazione.
I rapporti sentimentali e sessuali non devono danneggiare la produttività e per questo devono essere sottoposti a un rigido controllo.
\\Il denaro e l'erotismo sono sempre più legati e la prostituzione è il caso estremo della trasformazione dell'essere umano in oggetto, è un processo di alienazione e in seguito di commercializzazione dell'eros.
Anche Moravia conclude la storia di Dino con il tentativo da parte del protagonista di trasformazione della donna amata, ricoprendola di banconote, in puro valore economico.
Nel romanzo breve \textit{Agostino} il protagonista si chiede, dopo non essere riuscito ad accedere a un bordello, come sia possibile quantificare in denaro l'amore e riflette sull'aspetto sadico dell'atto di monetizzare una donna.
Dal 1943, anno di uscita di \textit{Agostino} agli anni Sessanta il rapporto tra forze economiche e sessualità mutò radicalmente.

Le promesse fatte dalla società consumistica e capitalistica riguardo la liberalizzazione della società non vennero mantenute: la sessualità venne banalizzata e inserita nelle leggi del mercato per poterla controllare e rendere innocua.
\\La letteratura non è contraria al progresso, ma ha l'obiettivo di dare voce ai mondi non allineati a esso, di formare un compromesso tra logica dominante e il represso.
Essa si presenta come luogo privilegiata di un discorso alternativo a quello del potere.

\paragraph{}Negli anni Settanta, come si è visto, il discorso sulla sessualità irruppe nel discorso pubblico, essa divenne il punto di maggiore interesse per l'indagine dell'individuo.
In letteratura, Pasolini escluso, fino a questa altezza storica l corpo e l'erotismo, anche se considerati un nucleo importante, furono inseriti nel sistema letterario solo sotto l'insegna dell'allusione e del sottinteso.
I nuovi autori, soprattutto i più giovani, vollero rappresentare la sessualità della loro generazione nella sua completa potenzialità, estranea a ogni sistema codificato.
\\Il nuovo gruppo etichettato con il nome di \textit{giovani narratori}\footcite{giovani} voleva raccontare il reale guardandosi attorno senza pregiudizi, riconoscere e rappresentare la nuova società.
Durante gli anni delle analisi apocalittiche di Pasolini e la fine delle contestazioni studentesche la nuova generazione tentava di promuovere l'idea di una letteratura giovanile in cui la sessualità non solo si trovava al centro, ma era esibita in modo esplicito.
\\Sessualità e ideologia sono difficilmente separabili per gli autori che avevano vissuto il Sessantotto.
Un'opera rappresentativa di questa nuova tendenza è \textit{Porci con le ali} (1976), scritto giovanile scritto da Lidia Ravera e Marco Lombardo Radice.
È il racconto di un'esperienza sessuale libera, che vuole mostrarsi svincolata dal sistema ma che si rivela non poter fare a meno di una codificazione.
\\La quotidianità doveva essere vista e descritta in modo diretto, con un linguaggio basso: la forma letteraria e poetica non doveva più deformare e sublimare la realtà.
I corpi dei giovani, abiurati da Pasolini, entrano nella letteratura in un'incondizionata esaltazione della loro espressività.

\paragraph{}La mercificazione dei corpi nella società capitalistica rappresentata da Pasolini in \textit{Salò} ebbe seguito nei lavori letterari degli anni Ottanta e Novanta.
\\È questa la generazione dei "cannibali".
Una caratteristiche di questa nuova narrativa italiana è quella dell'ampio uso di descrizioni di scene trasgressive, sanguinarie e di presentare aspetti centrali della vita di quegli anni come l'influsso della tecnologia, dei mezzi di comunicazione come il cinema e la televisione e le droghe pesanti.
Il sesso e la violenza vengono esibiti con insistenza, anche in modo esasperato e in certi casi anche comicizzati.
Il sesso si esplicita in immagini iperpornografiche con un grande fluire di liquidi corporei e la violenza di declina in azioni mi mutilazioni, sventramenti da cui consegue un largo scorrere di sangue.
Pornografia e crudeltà diventarono le rappresentazioni dell'eccesso di esibizione corporea.
Uno degli autori indicativi di questo movimento è Aldo Nove che con effetto grottesco riproduce le ossessioni della società: la pornografia, la violenza, la televisione.
I suoi personaggi agiscono come sotto effetto di una droga pubblicitaria che ne determina i comportamenti.

Se in Pasolini i giovani, ultimo baluardo di libertà, venivano sottomessi dal Potere con un'azione violenta, alla fine del secolo corpo e merce non sono più distinti, non c'è un'azione attiva di controllo, sono ormai parte di un unico flusso.
Il corpo ormai ridotto a puro oggetto non è più un soggetto da conservare, va anzi distrutto.
La sessualità diffusa attraverso le pubblicità ha perso ogni valore simbolico, il valore di denuncia presenta fino a pochi decenni prima è annullato.
\\I corpi tanto esposti non sono più luogo di lotta tra pulsioni e controllo o simbolo di liberazione rispetto agli schemi sociali, sono involucri vuoti, da squarciare e macellare.
\\È un estetica che porta al limite la rappresentazione della sessualità arrivando a disintegrare il corpo.

\paragraph{}Forzati i confini dell'espressione del corpo è necessario che esso assumi un nuovo valore, che si faccia portatore di un nuovo linguaggio.
È questo l'obiettivo degli autori più moderni: riempire nuovamente di significato il corpo svuotato.






\subsection{\textbf{\textit{Porci con le ali}}}
\textit{Porci con le ali}, romanzo in forma di diario scritto a quattro mani da Lidia Ravera e Marco Lombardo Radice, fece scandalo nel 1976 e allo stesso tempo ottenne un successo inaspettato.
Nonostante il successo la critica non si occupò di questa opera.
Venne considerato un romanzo per adolescenti di basso livello letterario, relegato al suo aspetto pornografico e alla rappresentazione, contestata, di una generazione.
Non venne valutato che dietro alla storia di due adolescenti si nascondono riflessioni sociologiche e culturali.
È un caso letterario.

\paragraph{}La recensione di Zircone dopo la prima pubblicazione, che prevedeva una tiratura di sole 6.000 copie, si rivelò profetica: \enquote{\textit{Porci con le ali} è un libro piuttosto bello e, potenzialmente un \textit{bestseller}. Probabilmente lo leggeranno in centomila, malgrado il sottotitolo scoraggiante\footnote{Il sottotitolo era: \textit{Diario sessuo-politico di due adolescenti}} […]. Sicuramente ne ricaveranno un film e molte tavole rotonde verranno dedicate ai vari problemi che passeggiano tra i suoi capitoli […]. Ne viene fuori una mistura molto affascinante, verissima, con i suoi difetti e le sue sbracature commerciali […]. Il libro è incantevole. In primo luogo per il linguaggio, spesso sporcaccione, mai fasullo, realistico fino alla crudeltà}\footcite{Zincone}.
\\\textit{Porci con le ali} vendette migliaia di copie in poche settimane, se ne parlò molto, ricevette critiche positive ma anche accuse di essere una mera operazione commerciale, a tratti pornografica.
Gli autori, in diverse occasioni, sottolinearono l'assenza di ricerca di successo e fama contrapponendo la loro volontà di contribuire alla discussione, alla riflessione critica e di  dare voce e corpo a esperienze dei giovani così da offrire materiale di dibattito.
È un successo che venne vissuto come un tradimento, una complicità con il nemico.

Nel giro di un solo anno venne prodotta una trasposizione cinematografica con Paolo Pietrangeli alla regia.
Il film vietato ai minori di 18 fu censurato per oscenità e poi fatto uscire di nuovo in una versione tagliata e vietata ai minori di 14 anni.
Nonostante fu accolto dalla critica in maniera negativo fu visto da mezzo milione di spettatori.
Ravera  e Lombardo Radice dopo un iniziale tentativo di partecipazione alla stesura della sceneggiatura presero le distanze dal prodotto finale.
L'autrice in un'intervista\footnote{È un'intervista guidata da Arbasino per il programma televisivo \textit{Match}, andato in onda tra il 1976 e 1977, che metteva a confronto due ospiti con personalità divergenti, Lidia Ravera era in confronto con Susanna Agnelli \url{ https://www.raiplay.it/video/2016/11/Susanna-Agnelli-e-Lidia-Ravera-c840bd18-9af6-4cf0-82c7-a7cb5907955d.html}} commentò che il lavoro di Pietrangeli enfatizzava troppo la \textit{noia della politica}.
Spiegò che effettivamente era presente il sentimento di insofferenza verso la politica nel momento in cui i giovani cui scoprono la coppia, l'amore e la sessualità, ma rimaneva una forte tensione politica verso la ricerca di un ruolo sociale ben definito.
Il rapporto con la politica per i ragazzi di questa generazione postsessantottina non è facile, ma un errore sottolinearne solo la noia e il disagio.

\paragraph{}Ci sono opere per le quali è più adatto premettere un discorso di tipo storicistico e altre per le quali un lavoro di questo tipo limiterebbe il loro valore.
\\Lidia Ravera è un'autrice con l'interesse di farsi interprete della sua generazione, della cultura giovanile, della violenza degli anni di piombo.
Ancora oggi si fa portavoce di coloro che negli anni '70 erano ragazzi e che ora sono adulti che devono fare i conti con il loro passato\footnote{significativi sul tema dei destini di vita di coloro che hanno vissuto gli anni di piombo sono opere come \textit{La festa è finita} (2002) e \textit{La guerra dei figli} (2009)}, negli ultimi anni presta particolare attenzione ai corpi della donne che invecchiano sviluppando il tema in chiave femminista e di emancipazione\footnote{si può fare riferimento a opere come \textit{Il terzo tempo} (2017) e \textit{Age pride. Per liberarci dai pregiudizi sull'età} (2023)}.

Un testo come \textit{Porci con le ali} (1976) necessita, forse, considerando anche l'interesse dichiarato dell'autrice, di entrambe le letture.
\\È un'opera che non può non essere messa in relazione con il contesto storico e culturale post sessantottino.
Rocco e Antonia partecipano a manifestazioni della sinistra extraparlamentare, discutono argomenti come il femminismo e il marxismo.
La rivoluzione dei costumi e della sessualità sono centrali nell'opera così come il complesso e in parte contraddittorio rapporto tra adolescenti e politica.
Nel corso delle pagine sono nominati personaggi politici come Berlinguer e riviste dell'epoca come \textit{Quaderni vicentini}, \textit{Panorama}, \textit{Crtica Marxista} e molti altri.
\\Lo stesso romanzo doveva inizialmente uscire, come spiega l'autrice nella prefazione nell'edizione del 2001, come un \enquote{pamphlet, un libello a circolazione interna (...). Non c'entrava l'idea di "pubblicare". C'entrava la politica (...), c'entrava quella gigantesca balera postsessantottarda  in cui tutto sembrava possibile, improbabile, e comunque doveroso}\footcite[7]{PrefazionePorci}.
Ravera spiega la scelta della forma diaristica proprio in virtù del rendere \enquote{gli anni Settanta protagonisti quanto il sesso, quanto l'amore, quanto la scrittura}\customfootcite[8]{PrefazionePorci}.
Il romanzo non vuole però avere pretese documentaristiche: la narrazione si concentra su dialoghi e processi di coscienza e autoriflessione delle figure principali in un breve periodo temporale.

Questa è però un'opera, pur non essendosi affermata come opera di prestigio letterario, la cui storia editoriale va oltre ai limiti del contesto autoreferenziale.
Il romanzo passò da una generazione all'altra continuando a offrire ai giovani spunti di riflessione.
\\La stessa autrice davanti a cifre di diversi milioni di lettori in tutto il mondo si chiede in prospettiva di una nuova ristampa a chi possa essere ancora indirizzata.
\\Si rivolge ai "figli" di Rocco e Antonio, ai nuovi giovani.
Non sono loro lettori, come lo erano i loro "genitori", arrabbiati davanti a una rappresentazione di quel tipo che non sentono propria, non grideranno di non essere così.
\enquote{È logico che i Rocco e Antonia d'oggi non riescano a vedere, nel penetrare l'uno il corpo dell'altra, nel darsi quella simbiosi breve del piacere, alcun sottotesto di battaglia, nessuna possibile bandiera}\customfootcite[9]{PrefazionePorci}, è una lettura diversa, fatta da persone distanti nel tempo e nello spazio dall'esperienza di due ragazzi degli anni Settanta, ma non per questa ragione meno valida.
\\Il fatto che ancora oggi sia un'opera attuale lo dimostra la decisione di riproporla in una recente l'edizione, del 2016, che dà una nuova forma al romanzo, quella della graphic novel.
Bisogna allora riflettere su come generazioni diverse possano sentirsi in qualche modo legate attraverso questo testo.

\paragraph{}La Roma del 1976 vissuta da due ragazzi non rappresenta più solo se stessa.
Anche se l'intento degli autori era di proporre il ritratto di una specifica situazione politica ed esistenziale non possono controllare il modo in cui i lettori recepiscono, e soprattutto recepiranno nel futuro, l'opera.
La realtà rappresentata perde la sua individualità entrando a far parte di una \textit{classe logica}\footnote{si fa riferimento alle riflessioni dello psicanalista cileno Matte Blanco e al principio di generalizzazione. Con classe logica si intende un insieme di elementi diversi che condividono almeno una qualità e che per questo per l'inconscio diventano equivalenti} più ampia che si potrebbe sintetizzare nel concetto di rapporto tra personale e pubblico.
In questo modo è possibile spiegare come lettori lontani culturalmente e cronologicamente dall'esperienza degli anni Sessanta e Settanta italiani abbiano potuto trovare interesse in un mondo a loro sconosciuto.
\\Attraverso il pensiero diurno, razionale e conscio, sembra impossibile che lettore si immedesimi in una situazione così precisa e specifica, ma l'inconscio recepisce l'universalità della vicenda.
Il turbamento tipico adolescenziale, la ricerca di un significato dell'esistenza e del proprio sé, la pressione del giudizio e delle aspettative altrui, la volontà di essere anticonformisti  si collocano al di fuori dell'esperienza singolare e individuale di due ragazzi degli anni '70.


\paragraph{}\textit{Porci con le ali} non è soltanto  la storia di Rocco e Antonio, piccolo borghesi romani nel 1976, non è solo il ritratto di una generazione, né il manifesto di una rivolta, ma è una riflessione sull'amore, sulla sessualità, sul sentimento di inadeguatezza, sul bisogno di libertà.


\subsubsection{Un'opera rivoluzionaria, il personale e il politico}
Il critico Gianni Turchetta notò che \enquote{gli anni Settanta [sono] a tutt’oggi il periodo della recente storia italiana meno rappresentato, e meno felicemente, dalla nostra letteratura}\footcite{Giovanianni70}.
È una fase storica che gli italiani tendono a sentire cronologicamente vicina, ma allo stesso tempo, forse a causa delle contraddizione che la percorrono, distante, un capitolo concluso e non ancora abbastanza lontano per essere discusso senza coinvolgimenti personali.
Ovvero, gli italiani non sembrano ancora in grado di accogliere le ragioni storiche, lo sviluppo è gli esiti nella loro contraddittorietà.

\textit{Porci con le ali} è uno dei tentativi di narrazione di questo periodo.
Scritto "a caldo", nel 1976, intende dare voce ai giovani degli anni '70, coloro che non hanno vissuto il '68 ma che devono fare i conti con i cambiamenti potati da quell'esperienza rivoluzionaria.
Gli autori di questo diario rappresentano nella loro scrittura a quattro mani un momento chiave della crisi della società italiana attraverso gli occhi di due adolescenti.
Non è più lo sguardo di un intellettuale a investigare la realtà, ma due ragazzi alle prime esperienze politiche, sociali e sessuali.
I giovani non hanno più un ruolo passivo all'interno della società, anzi forse non avendo vissuto in prima linea le vicende rivoluzionarie di anni Sessanta e Settanta sono i migliori interpreti delle contraddizioni della rivoluzione culturale e della nuova società capitalista.
\\Le vicende di \textit{Porci con le ali} ruotano attorno alla storia d'amore tra Rocco e Antonia, la trama è semplice.
I protagonisti, che si raccontano in prima persona, sono due studenti, minorenni, di un liceo romano e militanti di un gruppo di sinistra extraparlamentare.
Il racconto si alterna tra scene di riunioni pseudo-politiche e rapporti sessuali influenzandosi a vicenda. 

\paragraph{}Motivo centrale dell'opera è il rapporto tra pubblico, inteso come politico, e privato.
Il titolo stesso, che fa pensare all'improbabilità di una situazione dal momento che è impossibile che i porci abbiano le ali, sottolinea la relazione tra queste due sfere, come spiega Ravera: \enquote{le ali sono sicuramente nel comunque decidere di vivere la sessualità in ogni caso in modo problematico e politico cercando di capire di andare avanti}\footnote{dall'intervista nel programma \textit{Match} condotto da Arbasino \url{ https://www.raiplay.it/video/2016/11/Susanna-Agnelli-e-Lidia-Ravera-c840bd18-9af6-4cf0-82c7-a7cb5907955d.html}}.
\\Il personale che diventa politico, un punto fondamentale per il femminismo del Secondo Novecento, in questo romanzo risulta quasi capovolto.
Il politico entra più che mai nella vita privata dei protagonisti, non c'è azione non vista e vissuta in una prospettiva ideologica, anche l'amore non è libero da condizionamenti marxisti.
Sembra non esserci più spazio per agire in modo davvero libero.
La politica è una parte importante e ingombrante della vita dei due ragazzi.
Essi agiscono in funzione di essa, nella speranza, a tratti anche ai loro occhi irrealistica, di una rivoluzione erede del Sessantotto.
\\Gli anni Settanta presentati nell'opera che si presentano come l'epoca del \textit{collettivo} mostrano in realtà uno svuotamento del sentimento politico.
L'ideologia e l'attivismo si rivelano una gabbia che impedisce una piena e sincera espressione individuale: vivono con la paura di non essere abbastanza di sinistra, di non aver letto un articolo considerato fondamentale per la coscienza politica, di avere un'opinione diversa dal resto del collettivo.
Il bisogno di far sentire la propria voce è sostituito dalla paura di non essere omologato al proprio gruppo di riferimento, di essere inquadrato come un \enquote{intellettuale disorganico}\footfullcite[41]{Porci}.
Uscire dalla massa uniformata dove tutti si mostrano uguali agli altri è assolutamente da evitare.
L'omologazione è un principio di vita fondamentale nella generazione dei protagonisti.

\paragraph{}Non è un rapporto sereno quello di questi giovani con la politica, è vissuto con fatica e paura del giudizio.
Rocco teme  che gli venga domandato quante volte ha letto Gramsci o se ha letto l'ultimo saggio uscito su \textit{Luci gialle}.
Una rivista descritta come \enquote{una delle mie angosce quotidiane perché pare sia assolutamente geniale, fondamentale, scritta da compagni paraculissimi (...). Io ne ho comprato un numero solo, una volta che mi sentivo particolarmente volenteroso e intellettuale, e mi è sembrata una cosa da suicidarsi dalla noia}\customfootcite[42]{Porci}.
Non è quello di Rocco un interesse personale verso alcuni argomenti, è un dovere per poter essere un buon \textit{compagno}.

Nel flusso di pensieri privati di Rocco e Antonia non mancano numerose critiche nei confronti delle organizzazioni politiche di cui fanno parte, emergono sentimenti di disagio e rabbia nel non sentirsi sempre pienamente parte di un'ideologia tanto, forse troppo, discussa.
Gli incontri narrati nelle pagine sono spesso confusionari durante i quali vengono dette tante parole che non trovano mai una concretizzazione nella realtà.
Rocco è nello scrivere a Luca che riesce a essere sincero ed è in una lettera a lui indirizzata che commenta il gruppo di cui fa parte: \enquote{Grandi novità non ce ne sono, anzi ti dirò di più è esattamente lo stesso, con la differenza che mi sta passando l'entusiasmo e la voglia di cambiare qualcosa (...). Col gruppo ci troviamo di fronte di soliti casini di sempre (...). Strippiamo a turno (...) e poi siamo sempre al punto di prima. Forse non arriviamo mai al punto, o forse siamo troppo vigliacchi per affrontare i problemi veri, i casini di fondi}\customfootcite[45]{Porci}.


\paragraph{}Antonia, tra i vari personaggi, è quella che ha più consapevolezza delle contraddizioni interne alla loro vita politicizzata.
Disillusa prende coscienza anche della loro condizione di studenti: \enquote{Siamo due di cui parlano tutti, perché tutti parlano dei giovani, ma non parliamo mai. Non abbiamo diritto di parola. Ci spostano di qui e di lì, chiacchierando pomposamente dei nostri bisogni}\customfootcite[121]{Porci}.
Non hanno realmente una voce e il loro modo di fare politica, chiusi in un'aula parlando della guerra del Vietnam e della pace, fingendo di aver letto tutti i giornali di sinistra non cambierà la loro situazione, non renderà l'Italia un paese giusto e di sinistra.
È come se essere parte di un gruppo politico fosse una convenzione sociale o un tentativo di trovare un posto nel quale identificarsi in una società bombardata da immagini e mode e non più una necessità personale.

\paragraph{}Il rapporto che si instaura con la politica è in questo modo non basato sui giusti presupposti, non nascendo da un'autentica convinzione si traduce in contraddizioni difficili da sciogliere.
Rappresentative di questo dissidio esistenziale, tra necessità personali e doveri politici, sono le parole di Antonia: \enquote{Se lo volete sapere sono stufa di tutto questo chiacchierare senza dire un accidente. In tanto vuoto, il governo delle sinistre e le partite di pallone rischiano di assomigliarsi troppo. (...) Ma no, non è che a me freghi un cazzo della politica, è che se non mi aiuta almeno un po' a funzionare meglio, a capire perché sono cattiva e triste, guarda davvero non so che farmene. (...) E sarò una femminista di merda finché ti pare}\customfootcite[153]{Porci}.

È proprio nell'essere femminista che Antonia cercava di fondare la sua persona.
Rocco inizialmente la conosce solo nella sua identità politica di femminista e membro del collettivo e nei suoi costumi sessuali libertini, lei stessa non sa cosa aggiungere per descriversi meglio.
Antonia presentandosi a un uomo più grande con cui avrà dei rapporti sessuali ritiene importante fargli sapere di essere comunista e femminista, il resto non conta.
Il mostrarsi come la perfetta femminista è più importante di esprimersi in modo sincero.
La ragazza durante una riunione decide di non esprimersi a favore di Rocco per non \enquote{avere l'aria di quella che ragiona con la passera e si schiera sempre a fianco del suo "signore e padrone"}\customfootcite[83]{Porci}.
\\Il femminismo è capovolto, dal dare voce al sesso femminile ora sembra costringere Antonia a non parlare per paura di non essere più considerata parte del movimento.
Antonia non riesce più a riconoscersi in esso, ma neanche a trovare la sua identità al di fuori.
La libertà alla base del femminismo è soppiantata dalla necessità di conformarsi.
È interessante su come il femminismo era vissuta da delle giovani ragazze la narrazione di una serata tra compagne: quello che doveva essere un momento di dibattito sui diritti delle donne si trasforma in una chiara rappresentazione di un'ideologia che forse, seppur condivisa, non era vissuta in pienezza.
Per Cinzia secondo Antonia il femminismo è \enquote{una dannata via di scampo dalla solitudine e dai complessi}\customfootcite[136]{Porci} solo perché è grassa, le riunioni sono ormai diventate un\enquote{salottino da confidenze}\customfootcite[138]{Porci} nel quale lamentarsi dei fidanzati sessisti e maschilisti.
Commenta che \enquote{è inutile continuare a militare il nostro essere donne se poi il primo cazzo è un richiamo così irresistibile}\customfootcite[139]{Porci}.
\\Parlare di femminismo e di politica è una formalità, quasi un dovere per poter essere accettati.
È un processo che svuota la vera ideologia, la sua autenticità, e non può che tradursi in un fallimento.
In questo caso c'è anche una lettura falsata e rovesciata dei valori in cui i ragazzi dicono di credere e per i quali pensano di volersi battere.
Nell'esempio prima citato sembra impossibile la coesistenza del sentimento femminista e del desiderio sessuale, in altri casi invece lo stesso pensiero femminista spinge a dover avere determinate esperienze sessuali così da risultare totalmente emancipate.
È una continua contraddizione il cui risultato non è la libertà che i protagonisti del libro credono di avere.

È il quadro di una generazione che sta crescendo in una società che è mutata troppo in fretta.
Gli italiani, anche i più giovani, si trovano in difficoltà nel far convivere una lunga tradizione con i nuovi modelli e costumi.
Questo non significa che auspicano a un ritorno del passato, ma vivono dentro di loro un turbamento dovuto al fatto che la velocità con cui erano avvenuti i cambiamenti aveva creato delle contraddizioni non ignorabili.
La politica era un luogo d'incontro, dove trovare se stessi e ritagliarsi un posto nella società, ora con le sue convenzioni è diventata motivo di crisi esistenziale.

\paragraph{}La scena in cui emerge la sincera ideologia politica di entrambi i ragazzi, confermando la volontà a una maggiore giustizia sociale e non al ritorno della tradizione, è quella di una manifestazione di protesta contro la morte di un \textit{compagno} alla quale Rocco e Antonia partecipano soli, staccati dal collettivo e dalle convenzioni di esso.
Il sentimento verso l'ingiustizia della morte di una ragazzo ammazzato è vero e puro.
Rocco non crede di essere uguale a quel ragazzo, non crede di avere quel coraggio però l'idea che \enquote{un coglione di carabiniere ti spari addosso solo perché sei comunista e hai i capelli lunghi e vuoi riprenderti quello che è tuo, e per colpa di quel coglione e di chi ce l'ha mandato tu hai finito di mangiare, di fare l'amore, di bere, di andare al cinema, di fare il bagno al mare} lo fa \enquote{proprio strippare}\customfootcite[50]{Porci}.
Lui stesso scrive che per la prima volta aveva veramente voglia di andare a una manifestazione, non per vincolo sociale di mostrarsi contestatori a ogni costo.
Aveva addosso una rabbia enorme e \enquote{una gran voglia di dividerla con altra gente, di stare insieme ai compagni, di farglielo capire che un morto resta nostro ed è nostro}\customfootcite[51]{Porci}.
È questo il movimento emotivo che dovrebbe stare alla base dell'azione politica, la necessità di farsi sentire e di partecipare.
\\La rabbia di Rocco è condivisa da Antonia, in lei però davanti a un compagno che crede che il ragazzo abbia cercato la sua morte con le sue azioni e davanti alla polizia che le sembra quasi deriderli la rabbia si trasforma in sconforto.
Le sue convinzioni, il suo vivere in funzione della politica, del femminismo, leggere i giornali di politica, partecipare al collettivo non avevano più senso: \enquote{Di colpo io ero sola al mondo. In piazza non c'era più nessuno e di tutto quello che avevo fatto io niente era serio, niente era importante, niente contava, anzi non esistevano neanche né le mie idee né le mie azioni. (...) Mi è sembrato che qualsiasi regola (...) era inutile e stronza, insopportabile. Che quando muore qualcuno muore un pezzetto di te, ed è idiota far finta che tu continui a essere intero come prima}\customfootcite[53-54]{Porci}.
È in questi momenti che emerge l'autentico sentimento di questi due ragazzi di sinistra, in cui l'ideologia di sinistra esiste ed è sentita, ma che le convenzioni e la necessità di omologarsi agli altri militanti del collettivo spengono.

\paragraph{}La politica di questi anni è sempre più teorizzata, scritta su pagine di giornali e riviste, lontana dalla vita vera.
Non si parlava più di come risolvere i problemi nonostante la grande quantità di tempo dedicata a manifestazioni e incontri di stampo politico, le riunioni erano diventate il momento di grandi discorsi, di considerazioni teoriche.
L'agire è stato sostituito da un lavoro svolto sul piano teorico e filosofico, un'azione che non ha conseguenze sociali.
Una politica in cui si crede ma nella quale è difficile identificarsi crea nei giovani una spaccatura: da una parte si trova il bisogno di lottare per una società giusta e dall'altra lo sconforto nel vedere l'assenza di cambiamento.
È una politica destinata all'insuccesso.
È Antonia che attaccando Rocco rivela l'ipocrisia celata dietro alla loro militanza politica: \enquote{Non hai capito niente. Hai solo paura come tutti, e allora ti agiti e credi (fai finta) di essere uno diverso di uno che è com'è e non come lo fanno essere. (...) Già ma tu sei comunista solo al collettivo, dove ti ficchi le dita nel naso e guardi le gambe alle ragazze, ma ti senti comunque terribilmente di sinistra solo perché stai lì}\customfootcite[104]{Porci}.
La falsità è smascherata, ma più di tutto emerge la necessità di questi ragazzi di mostrarsi rivoluzionari, diversi dagli altri.
È un bisogno che si risolve, in modo del tutto contraddittorio ma giustificato dalla società consumistica di massa, nell'uniformazione e appiattimento dell'individualità.
In fondo questi ragazzi sono tutti uguali, omologati nel loro volere essere differenti.






\subsubsection{Lo stereotipo della sessualità libera}
\enquote{Cazzo. Cazzo cazzo cazzo. Figa. Fregna ciorgna. Figapelosa, bella calda, tutta puzzarella. Figa di puttanella.}\customfootcite[13]{Porci}, è così che si apre il romanzo.
\\A scandalizzare i lettori non fu soltanto la presenza di scene erotiche, anche omosessuali, ma anche la maniera realistica per descriverle scelta dagli autori.
Sembrano lontani i tempi delle fredde descrizioni di Moravia; un linguaggio esplicito, quasi volgare, e le descrizioni dettagliate presero il posto di perifrasi e di descrizioni simboliche.
La scelta di un linguaggio fortemente contaminato da espressioni volgari e contenente vari elementi fino ad allora poco consueti nell'ambito della produzione letteraria “ufficiale” è rivoluzionario forse come l'intento del romanzo di smascherare le contraddizioni della generazione degli anni '70.
Le vicende di Rocco e Antonia vogliono documentare in forma realistica i costumi mutati e per farlo c'è bisogno dell'utilizzo di una lingua nuova, libera da proibizioni e come era libera la loro sessualità.

A colpire però il lettore è come il linguaggio spregiudicato e i comportamenti che sembrano liberi da ogni obbligo e tabù in realtà celino un fondo di inautenticità e conformismo.
Quella che ha l'apparenza di essere una, finalmente, conquistata libertà sessuale e parità dei sessi all'interno di una relazione si rivela essere la maschera della necessità di sentirsi di sinistra, libertini.
È il bisogno di omologarsi nel gruppo di cui si fa parte.

La riflessione su come la politica fosse vissuta dai protagonisti, con l'eccezione di alcuni casi, in modo non sereno può essere estesa al loro modo di vivere la sessualità.
Si credono rivoluzionari ma forse sono solo, usando le parole di Pasolini, contestatori.
Credono di vivere la sfera affettiva e sessuale senza imposizioni e invece sentono il dovere di avere certi tipi di rapporti per dimostrare di essere di sinistra e, nel caso di Antonia, parte del movimento femminista.

Questa analisi non vuole contestare le conquiste del femminismo e della rivoluzione culturale e non riconoscere il fatto che la sessualità degli anni '70, quella di Antonio e Rocco, non fosse vissuta in modo diverso, con meno vincoli morali e più libertà, rispetto alla generazione dei loro genitori, ma vuole mettere alla luce alcune contraddizioni.
È un'incoerenza che potrebbe essere riassunta nell'affermazione che fa Antonia nei confronti del padre: \enquote{mio padre, comunista illuminato pronto a battersi per la liberazione sessuale delle polinesiane in lotta, pur di impedire quella di sua figlia}.

\paragraph{}\textit{Porci con le ali} è un interessante documento di un cambiamento di mentalità soprattutto nei confronti di esperienze sessuali omosessuali o del giudizio sulla perdita della verginità prima del matrimonio.
I giovani hanno superato la morale analizzata da Pasolini in \textit{Comizi d'amore}: l'amore gay e lesbico non è più una diversità non tollerata e l'arrivare vergini al matrimonio non è più un'opzione\footnote{queste osservazioni sono valide in un contesto urbano, metropolitano come quello di Roma}.
La contraddizione risiede nel fatto che avere rapporti da giovanissimi, anche di tipo omosessuali, dall'essere una conquista di libertà diventa un dovere.
La società nella quale vivono i due ragazzi sembra essere quella a cui auspicavano i giovani dei movimenti a fine anni Sessanta, ma tra le righe emergono i sentimenti di disagio e rabbia dei protagonisti.
Queste emozioni si rivelano soprattutto durante, o subito dopo, i rapporti sessuali dei ragazzi.

Controverso è il primo incontro privato tra Marcello e Rocco.
Marcello considerato dal collettivo \enquote{praticamente Dio}, uomo più grande la cui celebrità è basata sull'essere \enquote{membro di questo e di quello, compagno di questo che coordina quello}\footfullcite[35]{Porci}, spinge il ragazzo a parlare di masturbazione,  del suo rapporto con il pene, di esperienze omosessuali.
Rocco è imbarazzato, ma l'uomo, con tono di colui che dall'alto della sua posizione vuole istruire, spiega che l'imbarazzo è un sentimento borghese, così il giovane sente di non poter non rispondere: \enquote{mica mi posso sentire un residuo dell'ideologia borghese}\customfootcite[42]{Porci}.
Essere riservati riguardo la propria intimità è diventato qualcosa di condannabile.
Il discorso si trasforma in un rapporto omosessuale: Rocco è spinto ad accettarlo perché Marcello gli spiega che è giusto che ogni tipo di relazione venga sessualizzato.
\\Nel raccontare questo episodio a un amico Rocco ammette che teoricamente non dovrebbe sconvolgerlo l'avere avuto rapporti omosessuali, ma in realtà è molto turbato.
Il ragazzo vorrebbe essere in linea con le idee di Marcello, ma non riesce completamente: pensieri come il credere che il rapporto sessuale possa completare le relazioni sono alla fine solo \enquote{cazzate (...) su cui siamo tutti d'accordo (in teoria)}\customfootcite[46]{Porci}.
L'inciso in fondo alla considerazione di Rocco sottolinea come lui conosca la teoria, sa che secondo l'ideologia rivoluzionaria dovrebbe essere così, ma è più un'idea imposta che una conquista personale.
È d'accordo perché sente di doverlo essere. 
Vorrebbe essere realmente libero e vivere i rapporti con altri uomini, anche solo amici, come fossero la normalità, ma in realtà essi provocano in lui imbarazzo.
Il non riuscire a vivere il sesso come la "dottrina" del bravo attivista di sinistra che predica la libertà assoluta  gli crea disagio, un conflitto interiore tra ciò che realmente prova e ciò che vorrebbe provare.
\\Anche Antonia hai un rapporto lesbico con un'amica e di esso ricorda solo una sensazione di tenerezza mista a paura e rilassamento e non riesce a non vivere quella doppia masturbazione come una colpa e una vergogna.
È un'esperienza nuova, non riesce a elaborarla: \enquote{mi sembra di essere diventata di colpo un po' cretina}\customfootcite[143]{Porci}, ma è solo nei rapporti con altre donne che sente di voler stare lì per motivi diversi dalla sola paura di stare sola.

\paragraph{}Antonia, pur mostrandosi molto libera nel vivere la sessualità, vive i rapporti erotici con estrema freddezza, non sembra provare piacere, in più occasioni si sente usata, sente di essere considerata solo \enquote{un buco}\customfootcite[102]{Porci}.
Durante rapporti con un uomo più grande dice di sentirsi come un \enquote{capretto da macello}\customfootcite[27]{Porci}, pur avendo preso lei l'iniziativa, sta per piangere ed è nel mezzo di crisi d'identità.
Il sesso è per lei \enquote{o un'attività o una condanna}\customfootcite[27]{Porci}, non c'è nulla di piacevole, l'orgasmo viene infatti simulato, se non la \enquote{sensazione inebriante di potere: datemi un uccello in mano e solleverò il mondo}\customfootcite[27]{Porci}.
Il sesso viene usato come strumento di affermazione della propria esistenza nella società, una dimostrazione di femminismo e di potere.
Antonia sente di non avere voce, di non essere ascoltata come i suoi compagni e l'unica cosa che può fare è presentarsi come la ragazza carina che legge i \textit{Quaderni Piacentini} e avere rapporti sessuali con chi ha spazio nel dibattito politico.
Nulla di tutto questo è sintomo di libertà.



Rocco sembra invidiare la tranquillità di Antonia nel vivere la sessualità.
La ragazza gli spiega allora le sue insicurezza e lo fa con un discorso che rappresenta in modo limpido come una donna esista, ancora negli anni '70 nonostante le rivoluzione, solo nel momento in cui è riconosciuta come oggetto di desiderio.
Le sua parole sono queste: \enquote{Guarda che aver fatto l'amore, o averlo fatto senza eccessive paranoie non è poi questa meraviglia. Anzi le angosce ti vengono ancora di più (...). A me viene in mente che gli uomini mi usavo per fare tra le mie gambe le loro cose. (...) E allora mi sento più sola ancora, talmente sola che mi sembra quasi di non esistere. Per quello, magari giro per i corridoi con quella faccia che dici tu, (...) e cerco di farmi vedere graziosa e sensuale come la reclame di qualcosa da leccare. (...) Essere femmine è diverso: non è tanto aver fatto o non aver fatto, ma piacere o non piacere. (...) A me ogni tanto mi sembra di vivere solo per piacere agli uomini}\customfootcite[61]{Porci}.
\\La libertà sessuale di cui Antonia si mostra paladina in realtà nasconde il bisogno di essere accettata, vive con dolore la necessità femminile di essere: \enquote{intelligenti, emancipate e tutto il resto}\customfootcite[63]{Porci}
Antonia, giovane piccolo borghese, extraparlamentare scopre che la bellezza e il sesso per una donna sono merci di scambio, sono un modo per essere accettata e per ricevere in cambio, in un mercato triste, tenerezza e affetto.
Questo toglie alla ragazza il piacere della sessualità e la genuinità nel vivere una relazione.
\\Quando ha dei rapporti con Rocco vorrebbe soltanto sentirsi amata e invece in diverse occasioni si sente violata, vorrebbe si fermasse, che capisse che non sta provando niente se non vergogna e dolore.
Sono parole forti quelle che passano per la testa di Antonia mentre il fidanzato cerca di avere un rapporto anale: \enquote{Lo vuoi capire che mi sento violata se fai così! Ci ho messo due anni ad aprire le gambe senza avere paura (...). Odio sentirti alle mie spalle. E se lo vuoi sapere mi vergogno anche. (...) O, ti prego, ti prego, fermati, ma possibile che non capisci che non sento niente di bello}\customfootcite[95]{Porci}.
Tra i due manca completamente la comunicazione, Antonia non cerca di fermare Rocco e il ragazzo in realtà non vorrebbe ferirla in alcun modo.
Lei, si può forse ipotizzare considerando il modo in cui si presenta all'interno della società, ha paura di non mostrarsi disposta a ogni tipo di rapporto sessuale, di risultare "all'antica" e dall'altra parte Rocco non vuole mostrarsi insicuro, per essere uomo crede di dover essere \enquote{terribilmente virile e conquistatore}\customfootcite[62]{Porci}, quella in cui vive è una società \enquote{di cazzi duri}\customfootcite[63]{Porci}.

\paragraph{}Nelle descrizioni dei coiti, anche se così crude ed esplicite da sembrare rivoluzionarie, e nei pensieri di chi ne è coinvolto non si legge nessun grado di libertà; la sensazione è più di inquietudine misto a pena.
Questi ragazzi cresciuti da genitori che si mostrano estremamente di sinistra, a favore della sessualità libera, che hanno firmato per l'aborto e votato per il divorzio ma che poi scappano, come fa  la madre di Antonia, davanti a una conversazione sulla vita sessuale dei figli vivono l'emancipazione sessuale come un obbligo.
Questo perché la rivoluzione dei costumi ha imposto nuovi modelli ma gli italiani non sono riusciti a stare al passo con questi cambiamenti e si sono trovati a vivere in questa contraddizione.
I giovani credono davvero di essere loro la generazione che vive la sessualità senza vincoli, ma ogni loro azione si rivela essere un gesto simbolo di appartenenza, di omologazione.
I mutamenti degli ultimi decenni di rivelano dei cambiamenti gattopardeschi, apparenti ma non sostanziali.
Antonia e Rocco si accorgono di questo meccanismo che in realtà regola, invece di liberare, le loro esperienze.
Rocco inizia a sentire il bisogno di smettere di teorizzare perché si è \enquote{strarotto il cazzo di tutte le valanghe di teorie idiote e inconcludenti che sforniamo ogni mezzo minuto}\customfootcite[146]{Porci} e Antonia vuole una felicità \enquote{tutta diversa da quella che ci hanno proposto [al collettivo]}\customfootcite[104]{Porci}.
Auspicano a tipi di rapporti realmente nuovi, vissuti nella più completa libertà intesa come il fare solo ciò che si desidera, senza doversi sforzare per seguire determinati modelli o schemi. 



\subsubsection{Il manuale del rivoluzionario perfetto}

I giovani di questa generazione, quella che per prima cresce sotto i segni del consumismo e dell'omologazione parlano in modo quasi ossessivo di libertà, di scelta, di cambiamento, ma più che mai sono intrappolati in gabbie di generi e stereotipi e non hanno gli strumenti per ribellarsi.
I gruppi di ragazzi in questa opera sono nido di crisi e disperazione, ma anche culla del nuovo.
Il mito della rivoluzione sembra non avere più la presa di una volta su di loro, è sconfessato, legato al passato e quindi distrutto, ma rimane in qualche forma motore dell'esistenza del desiderio di rinnovare, capire e andare avanti.
Il mito della generazione precedente nelle sue convinzioni diventate convenzioni va rivisto perché quelle stesse convenzioni non siamo il motivo del fallimento delle azioni di rivoluzioni.

Il romanzo non vuole presentarsi caratterizzato da un discorso ideologico, è anzi quasi un rifiuto dell'ideologia.
Gli episodi del libro raccontano la realtà degli adolescenti degli ultimi anni Settanta: giovani che iniziano ad essere stanchi della politica e di un'ideologia che invece di liberare crea obblighi.
Si sentono scrutati nella loro vita privata e quotidiana, due sfere il cui confine è sempre più labile.
Le loro scelte personali sono influenzate dalla necessità di dare una determinata immagine pubblica, l’io è diventa un indefinito e corale noi.


\paragraph{}Se l'individualità si forma sull'essere un buon \textit{compagno} allora sono necessarie delle regole da seguire.
Ravera in un'intervista discutendo degli obblighi sessuali dei personaggi riferendosi ad Antonia fa riferimento al fatto che la sua generazione ha subito dei modelli di \textit{libertarismo autoritario}\footnote{l'intervista per il programma \textit{Match} \url{ https://www.raiplay.it/video/2016/11/Susanna-Agnelli-e-Lidia-Ravera-c840bd18-9af6-4cf0-82c7-a7cb5907955d.html}}.
Nel Sessantotto c'era il mito della non verginità, sentiva di doverla perdita.
Questo non è essere libero sessualmente, in questo caso particolare l'obbligo di arrivare vergini al matrimonio si trasformò nell'obbligo di perdere la verginità da giovani: è un vincolo che sostituisce l'altro.
Non è la libertà che veniva proclamata, è una libertà inquadrata in schemi all'interno dei quali è necessario stare.
Si forma così una sorta di manuale del perfetto rivoluzionario.

Antonia descrive Rocco come \enquote{uno di quelli che la rivoluzione la tirano fuori come il catechismo, come se fosse un insieme di precetti a cui attenersi, pena la dannazione rivoluzionaria}\footfullcite[79]{Porci}.
Non si è, dunque, rivoluzionari nel contrapporsi alle ingiustizie o nel battersi per dei cambiamenti, ma omologandosi a delle regole.
Queste norme influenzano non soltanto la vita pubblica degli individui, ma anche il tempo privato: Rocco insiste sulla necessità sua e di Antonia di non isolarsi dal gruppo perché sarebbe \enquote{una cosa borghese}\customfootcite[79]{Porci}, la loro intimità deve essere sacrificata nel nome della collettività.
Anche il modo di parlare di coppia deve adeguarsi al manuale di rivoluzione, bisogna rifiutare la coppia tradizionale, trovare forme non convenzionali, una donna non può parlare di casa, matrimonio, bambini e poi dichiararsi femminista.
\\L'attivismo politico deve prevalere sulle attività private.
Marco, ragazzo con il quale Antonia ha rapporti, vive l'amore come un lavoro e spesso si sente male perché in quei momenti \enquote{non era in riunione, non stava  prendendo accordi per un futuro impegno. Sul letto non riusciva a far salire le masse popolari e questo gli causava una specie di labirintite morale}\customfootcite[124]{Porci}.
Entra nel romanzo l'idea, già presente nell'opera di Bianciardi \textit{Integrazione}, di relazioni affettive e sessuali vissute come un lavoro, con orari prestabiliti.
Sono attività per le quali non si può perdere tempo e per questo vanno sottoposte a delle regole.
\\A essere sottoposto a una regolamentazione è anche il rapporto sessuale stesso, come se esistesse un manuale del sesso.
I pensieri di Rocco e Antonio rivelano che invece di godersi senza pensieri il momento del coito le loro teste sono piene di regole e convenzioni da seguire, l'ordine delle azioni da svolgere e le modalità.
Rocco prova vergogna nel fare \enquote{pensate da giornaletto porno}\customfootcite[90]{Porci}.
Questo rivela anche l'assenza di educazione sessuale, si predicava la libertà sessuale, ma non c'era un vero impegno istituzionale nell'istruire gli italiani sui temi relativi al sesso che continuavano dunque a restare un tabù in luoghi come le scuole. 


\paragraph{}La politica è ridotta a un libretto utopico e questo è un manuale spesso in conflitto con i desideri dei protagonisti.
I ragazzi si trovano a cercare di essere ciò che non sono, vivono con angoscia i loro pensieri \enquote{controrivoluzionari}\customfootcite[147]{Porci}
È un insieme di precetti che iniziano a stare stretti, ma che nonostante ciò i ragazzi cercano di seguire in modo rigoroso.
\\L'azione di inquadrare le rivoluzione in un manuale di istruzioni nasconde l'inquadramento della contestazione che perde, dunque, ogni valore rivoluzionario.
Se la rivolta viene integrata e addomesticata diventa impossibile, è ormai parte del sistema.
Come i film della \textit{Trilogia della vita} di Pasolini trasformati nell'apripista del cinema erotico italiano persero ogni spinta rivoluzionaria e di protesta così i giovani contestatori impegnati nel mostrarsi di sinistra più che esserlo erano stati addomesticati, non erano più pericolosi.
La vera rivoluzione sarebbe uscire dagli schemi di chi vuole presentarsi come rivoluzionario a tutti i costi.
Rocco e Antonia, almeno inizialmente, invece cercano a tutti i costi di mostrarsi conformi al gruppo extraparlamentare di cui fanno parte.
Fingono di leggere i giornali considerati di fondamentale importanza, partecipano alle riunioni in cui \enquote{si parla solo di scemenza, non si fa un discorso serio da cinque anni}\customfootcite[104]{Porci}, si comportano da bravi rivoluzionari partecipando alle manifestazioni, fingendo di vivere il sesso liberamente.

La sessualità è stata banalizzata, la libertà per cui si era lottato si è declinata in una nuova forma di controllo.
La definizione di Ravera di libertarismo autoritario è più che mai adeguata.
La parvenza di libertà è la maschera di nuovi obblighi.
L'assimilazione, anche se fasulla, delle rivoluzioni assicura la fine di ogni pericolo.
Se la repressione delle richieste di emancipazione e di libertà avrebbe causato sconvolgimenti di grande portata, l'integrazione fa in modo che essi vengano addomesticate e controllate.
È quella tolleranza \textit{grigia} che Pasolini a lungo cercò di spiegare e mostrare agli italiani, strappando quel velo che nascondeva l'autoritarismo della nuova società.
Sembra, forse con un parallelismo un po' forzato, riproporsi l'idea, in un contesto diverso, su cui si fonda il \textit{Gattopardo}: se la società si sta muovendo verso principi moderni di libertà e uguaglianza alle istituzioni conviene proporsi, anche fingendo, come guida del movimento perché \enquote{Se vogliamo che tutto rimanga come è, bisogna che tutto cambi}.
È quello che fa la società consumistica e capitalistica italiana nei confronti delle richieste rivoluzionarie.
Si mostra promotrice della libertà sessuale, del femminismo per assimilarli e annullarne la forza sovversiva.

\paragraph{}Sono ragazzi che vorrebbero cambiare il mondo\footnote{l'opera stessa si presenta con l'obiettivo di \enquote{cambiare la vita prima che la vita cambi noi}\footfullcite[7]{PrefazionePorci}}, ma sono schiavi del \textit{manuale del perfetto rivoluzionario}.
Iniziano però ad essere consapevoli delle contraddizioni interne alla loro generazione a differenza di quella dei \textit{capelloni} descritta da Pasolini.
\\Rocco, verso la fine, parlando del suo rapporto con Roberto a un amico rivela di non viverlo con vergogna come quello con Marcello, vuole sia spontaneo, teme invece la reazione dei compagni raggiunta la coscienza che \enquote{ a chiacchiere siano fortissimi su 'ste cose ma che di fronte a una storia vera non capirebbero un belino}\customfootcite[145]{Porci}.
Il collettivo non ha più l'influenza di prima su di lui, Rocco continua a frequentarlo, a essere di sinistra, ma vuole inseguire la possibilità di una relazione libera da vincoli che si possa evolvere su strade non predefinite, magari estranee a quelle previste dal manuale del buon ribelle.
L'erotismo che evade dalle regole è la vera rivoluzione.
\\Antonia invece rifiuta l'ideologia così come le era stata imposta, un insieme di precetti da seguire come dei dogmi religiosi.
Vuole fare la rivoluzione \enquote{perché le cose giuste fossero quelle che fanno stare bene}, non le basta l'ideologia per \enquote{consolarsi}\customfootcite[154]{Porci}.
Vuole diventare una rivoluzionaria vera, prova pena nel vedere, dopo la manifestazione della festa del lavoro,  un garofano che \enquote{poveraccio, anche lui, a fare il fiore da rivoluzione una volta l'anno}\customfootcite[154]{Porci}, non vuole che la sua militanza politica sia simbolica come il fiore che dopo essere stato usato nel giro di pochi giorni appassirà, vuole uscire dagli schemi controllati e cambiare realmente la società.

È in uno degli ultimi pensieri di Rocco rivolti ad Antonia che, forse, si intravede che dalla consapevolezza può nascere qualcosa di realmente rivoluzionario: \enquote{Anche le cose che a me sembravano chiare e limpide (...) in realtà erano confuse e contorte (...). E purtroppo l'unica cosa in cui avevi torto, era quando dicevi che per cancellare o scacciare queste cose bastano il femminismo, o i rapporti omosessuali, o la buona volontà, o la critica e l'autocritica, o la rivoluzione. E invece Antonia la mia grande angoscia di questi tempi è cominciare a vedere che tutte queste cose sono importanti, molto importanti, ma non sono ancora tutto, anzi sono forse solo una piccolissima parte di un viaggio molto molto lungo, che non so quanto duri né dove porti, e se porti da qualche parte. (...) Se non facciamo la rivoluzione non arriviamo da nessuna parte}\customfootcite[158-159]{Porci}.
Rocco non sa più quale sia il suo obiettivo, non sa quando sarà la rivoluzione, non sa come arrivarci, non ci sono più manuali da seguire.
Sa però che le pratiche che portava avanti con il collettivo non bastano più, i tempi sono cambiati e l'ideologia deve cambiare, non può cristallizzarsi, i principi ormai integrati, falsamente, dovevano trovare nuove spinte rivoluzionarie, tornare a essere pericolose.
\\Forse è finito il tempo dell'omologazione, del ribelle da manuale, forse Rocco e Antonio sono i primi a rendersi conto che fare il bravo comunista durante le riunioni, fingersi d'accordo, tenere \textit{Quaderni piacentini} sotto braccio, mettere i blue jeans, rifiutare la coppia solo perché è istituzionalizzata non è la strada giusta da percorrere perché avvenga la rivoluzione.

\paragraph{}Sembrava aver ragione Pasolini con la sua riflessione sul fallimento della rivoluzione, sulle mancate promesse della società dei consumi che predica libertà e genera nuovi obblighi morali.
I giovani che dovevano cambiare il mondo erano diventati tutti uguali, parte di una massa uniforme convinta di star facendo la rivoluzione e invece erano strumenti del nuovo Potere, convinti di fare politica avevano dimenticato le vere premesse perdendosi in teorizzazioni inutili.
Contava più presentarsi come un ragazzo di sinistra che avere la coscienza di un uomo di sinistra che poteva anche essere controcorrente con il pensiero dominante.
Aveva, inoltre, ragione nel parlare della liberalizzazione dei costumi come un'azione non autentica, la tanto sbandierata libertà sessuale si era rivelata davvero \enquote{una convenzione, un obbligo, un dovere sociale, un’ansia sociale, una caratteristica irrinunciabile della qualità di vita del consumatore}\footcite{Scritti9}.
\\Però come Pasolini vedeva nella bambina con le trecce le possibilità di un vero mutamento dei costumi forse Ravera e Lombardo Radice intravedevano nella generazione di Rocco e Antonio la consapevolezza necessaria perché le cose cambiassero davvero, non solo in apparenza.
\\Le contraddizioni erano state smascherate, una vita da rivoluzionario seguendo le regole non scritte del militante perfetto non era più praticabile e solo da questa nuova coscienza, dalla crisi esistenziale che ne era scaturita, era possibile fare la rivoluzione.


